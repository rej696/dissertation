\documentclass[../report.tex]{subfiles}
\graphicspath{{\subfix{../images}}}
\begin{document}

In recent years, the number of small satellites launched into low earth orbit
has been increasing, propelled by reduced launch costs and the expansion of the
private New Space industry. Flight software development for these satellites
must be completed in shorter time frames at lower costs for businesses to
remain competitive. Novel verification techniques are required to ensure
software can successfully be delivered on time and under budget, despite
constraints common in the industry such as reduced hardware availability.

Fuzz testing is a verification technique that has had success in identifying
critical vulnerabilities in general purpose software, such as server
applications. However, fuzz testing embedded software presents various
challenges, as these devices have more varied hardware architectures and
operating systems and reduced resources. The creation of tools and techniques
for embedded software fuzzing, such as rehosting based coverage guided fuzzing,
has been an active area of research.

In this work, several fuzz testing techniques are applied during the
development of an example minimal representative CubeSat flight software. The
work also implements a protocol grammar filter for the flight software
spacepacket communication interface, which is shown to improve the
effectiveness of rehosting based coverage guided fuzzing.

The process of developing flight software alongside a Unicorn based emulator
using blackbox and coverage guided fuzzing tools is demonstrated. The use of
emulation and fuzzing methodologies to discover and resolve issues is shown to
be effective. However, the implementation of the tooling necessary for these
techniques was found to be time consuming, and is likely not scalable to more
complex flight software development projects. Avenues for future research to
improve the applicability of these techniques, and reduce the development
effort required, are outlined.

\end{document}
