\documentclass[../report.tex]{subfiles}
\graphicspath{{\subfix{../images}}}
\begin{document}

% literature-review
% [ ] Literature Review of existing Embedded Emulation/Fuzzing Techniques
%      - Fuzzware, Hoedur
%      - Fuzzing Protocols with Fuzzware/Hoedur, introduce grammar filter?
%      - Papers don't emulate/fuzz rtos behaviours?


\section{Software Engineering in the New Space Industry}
% References:
% - Bousedra_2024: policy paper, may be less useful?
% - Denis_2020: paper discussing trends in the space industry, more relevent?
% - Cubesat_Handbook: book with information about designing cubesats, useful for satellite architecture?
% - Cratere_2024: OBC for Cubesats State of the Art

In the past, developing and launching satellites was the domain of large
corporations and governments due to the costs involved. Recently, In order to
take advantage of cheaper access to space (\autoref{sec:motivation}), a new
type of organisation has begun to build and operate space missions.

% TODO introduce characteristics of new space (low earth orbit, cubesats)

\citet{Sweeting_2018} defines New Space as
\begin{quote}
    The emergence of a different ethos for space where the established
    aerospace methods and business have been challenged by more entrepreneurial
    private sector by adopting more agile approaches and exploiting the latest
    commercial-off-the-shelf technologies
\end{quote}

\citet{Denis_2020} describes the current market trends of New Space as
"Darwinism in Space". The creation of new markets in the space environments as
a result of the digital transformation has led to a space industry "gold rush",
with large venture capital investment and a "winner takes all" incentive
driving high levels of competition \citep{Denis_2020}. The impact of this is
that New Space companies must design and manufacture more satellites at a
faster rate and for lower cost if they wish to be competitive.

As a result of the business environment, many New Space start ups have
introduced silicon valley concepts into their businesses, such as "DevOps
culture" and launching Minimum Viable Product (MVP) CubeSats. Furthermore, as
these companies transition from start-ups to mature businesses with an aim to be
profitable, LEAN practices become prevalent, resulting in attempts to reduce
any "wasteful" or expensive practices \citep{Denis_2020}. These market
conditions have led New Space businesses to become more accepting of risk than
their Big Space counterparts \citep{Cratere_2024}

Therefore, Flight Software needs to be developed in shorter time frames and with
lower costs. Satellites in low earth orbit have a shorter lifetime due to
atmospheric drag. CubeSats have a generally high failure rate due to a long
list of possible failure modes which can cripple a platform, with no chance of
repair. Some examples might be a solar panel not deploying, a launch vibration
damaging a critical sensor, or an issue with the thermal management resulting
in the payload having degraded operations. Often, the correctness of the
software systems only becomes an issue when all these barriers have been
passed, and so is a risk that is not considered worth spending time and money
until later in a programme. Thus, new space companies have a vastly different
approach to software systems compared with big space companies, and many of the
approaches used by big space companies to ensure software correctness, such as
extensive system and integration testing and formal methods, are simply to time
consuming and expensive to be applicable in the new space environment.

\section{Satellite Architecture}
% (SPAD book?, Cubesat Handbook)

Software systems are an essential component of modern satellite designs.
Satellites usually include a number of distributed embedded systems, each
tasked with monitoring and controlling different critical subsystems such as
power, attitude control, communication etc. Typically, a satellite will have a
central on board computer (OBC), or command and data handling subsystem (CDH).

There may be a dedicated guidance navigation and control (GNC) or attitude
determination and control (ADCS) subsystem which runs complex algorithms using
data retrieved from sensors like star trackers, rate sensors, magnetometers and
sun sensors to drive actuators like reaction wheels and magnetorquers in order
to point the satellite. Pointing is critical for many tasks. Payload operation
tasks such as imaging require a high degree of accuracy and responsiveness.
Propulsion manoeuvres, managing thermal loads, power generation (i.e. pointing
the solar panels at the sun to improve the efficiency) all require an effective
ADCS subsystem.

The OBC will be connected to many lower power telemetry (TM) nodes, which are
tasked with monitoring temperatures, voltages and currents, and toggling switches
to enable/disable different subsystems. The OBC will log the telemetry reported
by the TM nodes, and if telemetry exceeds limits, handle fault detection,
isolation and recovery (FDIR).

The OBC or CDH subsystem is responsible for handling communication with a
ground station via a radio, and so provides the interface between a satellite
and the operator. This interface is commonly referred to as Telemetry and
Telecommand (TMTC). The operator uses the TMTC interface on the OBC to download data
from orbit, schedule payload operations, and respond to issues.

Payload computers (PLC) are common on satellites with complex payloads and are
responsible for managing and operating whatever instruments are on board for
delivery of the mission, such as radios, imagers, scientific experiments etc.
On smaller satellites, like CubeSats where payloads are less complex, the
responsibilities of the PLC can be handled by the OBC.


\section{Flight Software Development Challenges}

% Discuss Nasa's software development practices, power of ten etc.
% Code Reuse and COTS parts.
CubeSats projects tend to make more use of COTS hardware. However, software that
meets the system requirements often cannot be purchased off the shelf, and so
development costs shift from hardware to software
\citep{Cubesat_Handbook_OBSW}. However, there are several flight software
frameworks, such as NASA's Core Flight System (cFS) \citep{Nasa_cFS}, or Bright
Ascensions Generation One (Gen1) Flight Software Development Kit (FSDK)
\citep{Bal_FSDK}. The main aim of these frameworks is to reduce the cost and
development time for Flight Software through code reuse and hardware
abstraction.

For example, a cFS application uses software abstractions of the operating
system and hardware, called the Operating System Abstraction Layer (OSAL) and
Platform Support Package (PSP). This allows correctly written application
code to easily be ported between different hardware architectures such as Arm
Cortex-M or SPARC-V8, and different operating systems like FreeRTOS, RTEMS or
Linux \citep{Nasa_cFS}.

The Gen1 FSDK provides libraries of pre-defined configurable software
"Components", which satisfy common Flight Software requirements, such as
logging telemetry. These pre-defined components can be composed together with
bespoke components into a "Deployment" which can be compiled for several
different COTS hardware options, such as the AAC-ClydeSpace Kryten, or the
GomSpace Nanomind \citep{Bal_Options}.

\citet{Farges_2022} evaluates the use of Software Frameworks in the space
industry. They find a number of quality related benefits, such as increasing
reliance on standardised communication interfaces like CCSDS Space Packet
Protocol; as well as cost related benefits attributable to code reuse between
satellite missions. However, they also identify risks with relying on software
frameworks, such as potential issues with licenses, and the limited
adaptability of software that is highly coupled to a framework, as implementing
functions that aren't currently supported in the framework can be difficult.

% FIXME: ( SpaceX's StarLink satellite program is the
% obvious example of this. These satellites have a short mission life of five
% years [REF], )

Developing flight software for satellite systems is a complex process due to
the large number of different interfaces and interdependencies in the
architecture. Furthermore, due to the space environment, computers used in
satellite systems can have stringent environmental requirements, such as high
radiation tolerance to reduce the likelihood of single event upsets (SEU), low
power consumption and consideration of thermal management. The embedded
microcontrollers that meet these requirements can be more expensive than those
commonly used for other applications, like in automotive systems or IOT devices.

Developing software for embedded systems often has a critical dependency on
hardware. Writing drivers for devices using only a datasheet can result in long
periods of rework when hardware is available to test against due to incorrect
information or a misinterpretation. The number of interdependent systems, the
costs of these components, and smaller budgets of new space companies, results
in dedicated engineering hardware required for effective development of
embedded flight software being unavailable. Often engineers are only being able
to test with flight hardware, and then have to fit their testing into a packet
schedule to meet tight deadlines.

% * FIXME should probably talk about protocols?

\section{Fuzz Testing}

% Problem Description from prep report
% FIXME rework to be more about how fuzzing could be a lower cost automated solution to system testing?

% CUT? Reword for space software? security concerns attack surfaces in space?
% Embedded systems are becoming more prevalent in today's society, due to the
% rise of the Internet of Things (IOT) \citep{Abdumohasan_2021}. Embedded
% software is often written in low level languages such as C where errors and
% security vulnerabilities are easy to introduce \citep{Svoboda_2021}, and have
% many many attack surfaces where these mistakes could be exploited
% \citep{Abdumohasan_2021}. Furthermore, many embedded systems have stringent
% reliability requirements, such as safety critical software in automotive
% devices. As such, verification and testing of embedded systems is paramount to
% prevent the presence of security exploits and critical bugs in these devices.
% CUT?

Many software programs use a methodology called fuzz testing to automatically
verify software. Fuzz testing is typically used to identify and resolve
security vulnerabilities for programs running on general purpose computers
\citep{Google_2023}. In its most basic form, fuzz testing, or fuzzing, consists
of generating some input test data and monitoring the response of the
software/system under test (SUT) to this input. If the fuzzer detects the
program has crashed, it saves the generated test data for analysis by an
engineer, and mutates the input data in some way to produce a different result.
This brute force approach is known as black-box fuzzing, and given enough time
to execute, will uncover many exploitable bugs with the system, such as memory
leaks, buffer overflows, and off by one errors. Basic black-box fuzzing on an
embedded system will require some fuzzing harness \citep{Eisele_et_al_2022} and
some method of detecting a fault, such as performing a liveness check
\citep{Yun_2022}. Borsig et al. do not consider black-box fuzzing as a
promising methodology for fuzzing an esp32 based IOT device, and conclude that
white-box and grey-box fuzzing techniques are more capable \citep{Borsig_2020}.

Compared to black-box fuzzing, white-box fuzzing involves targeted test case
generation through static program analysis techniques, generating a test case
to match every statically identified code path. Microsoft have had success
using white-box fuzzing to identify difficult bugs that were missed by
black-box fuzzing \citep{Godefroid_2012}. However, symbolic execution
techniques used in white-box fuzzing can become unfeasably computationally
expensive for larger projects \citep{Krishnamoorthy_2010}.

Grey-box fuzzing is a middle ground between black-box and white-box fuzzing.
Grey-box fuzzing relies on the fuzzer receiving feedback regarding code
coverage for each generated test-case to mutate and generate further test
cases. This approach allows it to find code paths faster than black-box
testing, but without requiring any static code analysis \citep{Yun_2022}.

Grey-box fuzzing of a program can be easily implemented for application (i.e
non-embedded) software using popular fuzzers like AFL. AFL provides a library
that is linked into the program at compile time, along with several sanitisers,
to instrument the program. This allows AFL to receive coverage information from
the running program during fuzzing \citep{AFL_2019}. Embedded systems often run on
constrained environments, where running the fuzzer on the target hardware would
be unfeasible. As such, the fuzzer and the SUT are run in different
environments, and it is difficult to instrument and forward runtime information
for grey-box fuzzing \citep{Muench_2018}.

Attempts to solve this problem focus either on fuzzing a program running on
target hardware, or in an emulated environment \citep{Eisele_et_al_2022}.
Running in an emulated environment allows easy inspection of the SUT by the
fuzzer, and potentially faster running and parallel tests
\citep{Eisele_et_al_2022}. However, emulating embedded systems correctly is a
hard problem, and due to the diversity in operating systems, hardware,
peripherals etc. often means that development effort spent on emulating one
system cannot be easily transferred to another. Attempts have been made to
automate emulator development, such as Clements et al. HALucinator, which
allows Hardware Abstraction Layer (HAL) functions in a compiled binary to be
stubbed and fuzzed \citep{Clements_2021}.
Chen et al. investigate a novel approach to fuzzing Real Time Operating Systems
(RTOS) which involves slicing a single program into its constituent tasks and
fuzzing them separately on an emulator based on the call graph of the program
\citep{Chen_2022}.

Yun et al. note that most embedded systems fuzzers rely on emulation
\citep{Yun_2022}. However, most bugs found through emulation still need to be
validated on hardware, and so being able to run the fuzz tests directly on the
target hardware is preferable \citep{Eisele_et_al_2022}. Several methods have
been discussed to improve instrumentation and feedback from running fuzz tests
on target hardware. Beckmann et al. propose making use of the tracing
facilities on modern Arm Cortex-M microcontrollers to stream coverage
information back to the fuzzer \citep{Beckmann_2023}. Eisele proposes using a
debugger as a method of inspecting the SUT and providing coverage feedback to a
fuzzer \citep{Eisele_2022}.

During the development of an embedded system, engineers often require access to
representative development hardware to effectively design and write software.
Often, waiting for prototype hardware to start software development does not
align with business deadlines. The use of emulators improves enables engineers
to write embedded software without access to development hardware, improving
productivity and reducing development time.

Existing research into embedded systems fuzzing often focuses on its
cybersecurity applications and benefits. However, fuzz testing can also be an
effective method for general verification of software. AdaCore provide a fuzz
testing tool called GNATfuzz, which enables subprogram level fuzz testing as a
supplement to unit testing for Ada and SPARK embedded software
\citep{gnatfuzz}. By isolating subprograms and building isolated fuzz test
executables in a manner similar to a unit testing framework, GNATfuzz removes
the need to be able to run the full program under a fuzzer, and thus the need
for an emulator or target hardware. However, the effectiveness of this approach
is reliant on Ada's extended runtime constraint checking when compared to C
\citep{gnatfuzz}. Furthermore, subprogram level fuzzing is less able detect
more complex issues resulting from interactions with hardware peripherals and
state.

While research into the fuzz testing of embedded systems has been increasing
year by year \citep{Yun_2022}, there are few generic solutions
\citep{Eisele_et_al_2022}. The two main challenges to embedded system fuzzing
when compared to desktop or server systems remain the variety in CPU
architectures in embedded systems, and the lack of an operating system such as
linux for bare metal systems \citep{Eisele_et_al_2022}. Currently, the
availability and ease of use of fuzzing tooling for embedded systems does not
match that of desktop applications.

% emulation options according to Chen/Yun/Eisele?

% Clements 2020 - Halucinator

% Scarnowski 2020 - Fuzzware

% Scarnowski 2023 - Satellite Fuzzing with Fuzzware

% Scharnowski 2020 - Hoedur

% Seidel 2023 - Faster Fuzzers

% Willbold 2024 - security analysis of satellites with fuzzers

% Zhang 2024

% Talk about the promise of rehosting/emulation/
% IOTFuzz
% HALucinator
% FuzzWare
% Hoedur
% Faster Fuzzers (near Native?)
% unicorn?

% Outstanding problems for firmware fuzzing?

% CUT? TODO reword to remove NRF52 and make generic about embedded fuzzing
The design and implementation of embedded software and its verification is
highly coupled to hardware. The nordic semiconductor NRF52 family of
microcontrollers are a common choice for IOT devices. NRF52s are based on the
ARM Cortex-M architecture, and include embedded radio peripherals, such as
bluetooth low energy and wifi (amongst others). Another common IOT radio
microcontroller is the ESP32, for which Borsig et al. previously developed a
fuzzing framework \citep{Borsig_2020}. Furthermore, Yun suggested the need for
further research into emulation of specific architectures and devices
\citep{Yun_2022}. An NRF52 was used by Beckmann for on-target coverage guided
fuzzing of a bare metal ARM Cortex-M system using instruction tracing over a
single wire output (SWO) interface \citep{Beckmann_2023}. Additionally, Behrang
outlines a method for using the unicorn cpu emulator to execute NRF52 based
bare-metal radio firmware \citep{Behrang_2023}. The desktop fuzzer AFL++
includes an implementation called "unicorn\_mode" which allows the unicorn
engine to be built with AFL++ support \citep{UnicornMode}. Maier develops the
Unicorefuzz framework and presents an example use of AFL++ Unicorn Mode for
fuzzing kernel modules, which have similar challenges to embedded systems
\citep{Maier_2019}.
% ENDCUT?

\section{Fuzz Testing in the Space Industry}
% Introduce the Space Context? (maybe earlier) reference willbold and gutierrez

% ** Flight Software security and protocols
% ...spacepacket, framing, checksums etc? Fuzzware and Hoedur

% Build on Hoedur and Willbold to integrate a protocol fuzzer for spacepacket/ccsds protocols?
% - investigate Hoedur, how does it work, how can it be extended?
% - investigate protocol fuzzers, what do we need to do to build one?
% - test against Willbolds open source satellite software with the protocol fuzzer, and compare results?

% CUT?
% However, while much research outlines the
% challenges with current embedded systems fuzzing techniques, it is unclear how
% these challenges compare with the effort required to verify embedded software
% using traditional techniques. In order to prove that embedded fuzz testing is a
% viable alternative to existing verification methods, a comparison of the cost
% effectiveness of conducting embedded fuzzing with other verification techniques
% is required.
% ENDCUT


\end{document}
