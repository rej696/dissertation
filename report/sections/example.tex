
% Examples

\section{Example Section}

Like all chapters, it will have a number of sections \dots

%-------------------------------------------------------------------------------
\subsection{Example Subsection}
\dots\ and subsections \dots

\subsubsection{Example Sub-subsection}
\dots\ and sub-subsections.

\section[Short Section Title]{Another Section With a Long Title and Whose Title Is Abbreviated in the Table of Contents}

%-------------------------------------------------------------------------------
\begin{table}[htb]
\caption{An example table}
\bigskip
\begin{center}
\label{Example-Table}
\begin{tabular}{|l|l|}
\hline
Items & Values \\
\hline
\hline
Item 1 & Value 1 \\
Item 2 & Value 2 \\
\hline
\end{tabular}
\end{center}
\end{table}

Another section, just for good measure. You can reference a table, figure or equation using \verb|\ref|, just like this reference to Table \ref{Example-Table}.

%%%%%%%%%%%%%%%%%%%%%%%%%%%%%%%%%%%%%%%%%%%%%%%%%%%%%%%%%%%%%%%%%%%%%%%%%%%%%%%%
\section{Example Lists}

%-------------------------------------------------------------------------------
\subsection{Enumerated}

\begin{enumerate}
\item Example enumerated list:
  \begin{itemize}
  \item a nested enumerated list item;
  \item and another one.
  \end{itemize}
\item Second item in the list.
\end{enumerate}

%-------------------------------------------------------------------------------
\subsection{Itemised}

\begin{itemize}
\item Example itemised list.
  \begin{itemize}
  \item A nested itemised list item.
  \end{itemize}
\item Second item in the list.
\end{itemize}

%-------------------------------------------------------------------------------
\subsection{Description}

\begin{description}
\item[Item 1]First item in the list.
\item[Item 2]Second item in the list.
\end{description}
