\documentclass[../report.tex]{subfiles}
\graphicspath{{\subfix{../images}}}
\begin{document}

In order to ensure the research questions outlined in \textbf{\autoref{sec:rq}}, a set of requirements needed to identified to drive the
design of any software needed.

% Requirements and Design
\section{User Requirements} \label{sec:user-req}

\textbf{\autoref{tab:rq1-reqs}} shows user requirements derived to satisfy
\refrq{1}. This requirement set shows the developments fundamental to the
project, namely the development of flight software (\refreq{RQ1-1}), emulator
(\refreq{RQ1-2}) and coverage guided fuzzer (\refreq{RQ1-3}), as "Must"
requirements. The development of a blackbox fuzzer was deemed to be a "Should"
requirement (\refreq{RQ1-6}, \refreq{RQ1-7}). Running fuzz tests on the
hardware (\refreq{RQ1-5} and \refreq{RQ1-8}) were identified as "Could"
requirements, as potential extensions to the work.

\begin{table}[H]
    \centering
    \begin{tabular}[c]{|l|p{10cm}|}
        \hline
        \definereq{RQ1-1} &
        A target software program must be developed that is representative of space flight command and data handling software
        \\
        \hline
        \definereq{RQ1-2} &
        An emulator must be developed that is able to execute the target software on an x86\_64 linux computer
        \\
        \hline
        \definereq{RQ1-3} &
        A coverage guided fuzzer must be implemented to fuzz test the target software
        \\
        \hline
        \definereq{RQ1-4} &
        The coverage guided fuzzer must be able to fuzz the target software running on the emulator
        \\
        \hline
        \definereq{RQ1-5} &
        The coverage guided fuzzer could be able to fuzz the target software running on the target hardware
        \\
        \hline
        \definereq{RQ1-6} &
        A blackbox fuzzer should be implemented to fuzz test the communication interface of the target software
        \\
        \hline
        \definereq{RQ1-7} &
        The blackbox fuzzer should be able to fuzz the target software running on the emulator
        \\
        \hline
        \definereq{RQ1-8} &
        The blackbox fuzzer could be able to fuzz the target software running on the target hardware
        \\
        % \hline
        % \definereq{A} & Should &
        % \\
        % \hline
        % \definereq{B} & \textit{Could} &
        % \\
        % \hline
        % \definereq{C} & \textbf{Must} &
        % \\
        \hline
    \end{tabular}
    \caption{User Requirements to satisfy Research Question 1}
    \label{tab:rq1-reqs}
\end{table}

\textbf{\autoref{tab:rq2-reqs}} shows user requirements derived to satisfy \refrq{2}.
This requirement set has some overlap with those identified to satisfy
\refrq{1} in \textbf{\autoref{tab:rq1-reqs}}. Both require some target software
to be developed (\refreq{RQ1-1}, \refreq{RQ2-1}) and both require the
implementation of a coverage guided fuzzer (\refreq{RQ1-3}, \refreq{RQ2-2}).

\begin{table}[H]
    \centering
    \begin{tabular}[c]{|l|l|p{10cm}|}
        \hline
        \definereq{RQ2-1} &
        A target software program must be developed that implements a communication protocol
        \\
        \hline
        \definereq{RQ2-2} &
        The communication interface of the target software must be representative of a space flight command and data handling system
        \\
        \hline
        \definereq{RQ2-3} &
        A coverage guided fuzzer must be implemented to fuzz test the communication interface of the target software
        \\
        \hline
        \definereq{RQ2-4} &
        A protocol grammar filter must be developed that ensures a random bytestream satisfies the grammar of the protocol
        \\
        \hline
        \definereq{RQ2-5} &
        The protocol grammar filter must be able to be integrated with the coverage guided fuzzer
        \\
        \hline
    \end{tabular}
    \caption{User Requirements to satisfy Research Question 2}
    \label{tab:rq2-reqs}
\end{table}

% TODO how will both these requirements sets satisfy the research questions?

\section{Derived Requirements} \label{sec:derived-req}

The user requirements outlined to satisfy the research questions in
\autoref{sec:user-req} identified three main pieces of software that needed to
be developed. A representative flight software (\refreq{RQ1-1},
\refreq{RQ2-2}), some emulation software capable of being used as a test
harness for a coverage guided fuzzer (\refreq{RQ1-2}, \refreq{RQ2-3}), and some
software for generating some valid communication according to a protocol
grammar (\refreq{RQ2-4}).

Additional requirements for each of these programs, namely the Flight Software
(FSW), Emulation (EMU) and Protocol Grammar Filter (PGF), were derived from the
user requirements and are outlined and discussed below.

%% TODO how to show what was derived from where? a seperate table after defining the requirements?

%% FSW Requirements
% Target software must be implemented to target arm cortex m
% Target software must implement the CCSDS spacepacket protocol as a communication interface
% The communication interface should include a packet verification method, such as a checksum or CRC
% The target software must allow the remote triggering of procedures
% The target software must allow the retrieval of telemetry data
% The target software could allow the setting of configuration data
% The target software could allow the retrieval of configuration data
% There must be a method to communicate with the target software from an x86\_64 development computer
% The target software should implement functionality similar to a real time operating system
%% TODO critique these requirements and provide rationale, showing how they will satisfy the research question

\begin{table}[H]
    \centering
    \begin{tabular}[c]{|l|p{10cm}|}
        \hline
        \definereq{FSW-1} &
        The target software must be implemented to target an ARM Cortex-M microcontroller
        \\
        \hline
        \definereq{FSW-2} &
        The target software must be implement the spacepacket protocol over a hardware communication interface
        \\
        \hline
        \definereq{FSW-3} &
        The target software should include a method of verifying packets received over the communication interface, such as a checksum
        \\
        \hline
        \definereq{FSW-4} &
        The target software must allow the remote triggering of procedures
        \\
        \hline
        \definereq{FSW-5} &
        The target software must allow the retrieval of telemetry data
        \\
        \hline
        \definereq{FSW-6} &
        The target software could allow the setting of configuration data
        \\
        \hline
        \definereq{FSW-7} &
        The target software could allow the retrieval of configuration data
        \\
        \hline
        \definereq{FSW-8} &
        There must be a method to communicate with the target software from an x86\_64 development computer
        \\
        \hline
        \definereq{FSW-9} &
        The target software should implement functionality similar to a real time operating system
        \\
        \hline
    \end{tabular}
    \caption{Derived Requirements for the target flight software}
    \label{tab:fsw-reqs}
\end{table}

\autoref{tab:fsw-reqs} shows the derived flight software requirements.
\refrq{2} is focused on fuzz testing the communication interface with the
flight software, and so several of these requirements clarify that interface.
The use of spacepacket protocol (\refreq{FSW-2}), the implementation of RTOS
functionality (\refreq{FSW-9}), and the need for remote procedure triggering
(\refreq{FSW-4}), are all identified to ensure the FSW follows standard FSW
design decisions (outlined in \autoref{chap:lit-rev}), as required by
\refreq{RQ1-1}.

Other requirements, such as \refreq{FSW-8}, are included to ensure that the
design of the FSW is easy to fuzz test, and thus satisfy both \refrq{1} and
\refrq{2}.

\refreq{FSW-1} specifies that the target software be designed to run on an ARM
Cortex-M microcontroller. This microcontroller architechture is very common in
space applications \citep{Cratere_2024}, so also helps to satisfy \refreq{RQ1-1}.

\autoref{tab:emu-reqs} outlines derived requirements for the emulator. Some of
these requirements are directly derived from the User Requirements, such as
\refreq{EMU-1} and \refreq{RQ1-2}. Others are fundamental for the emulator to
be able to be used to answer \refrq{2}, such as \refreq{EMU-6}.

\refreq{EMU-5} aims to enable the emulator to be used to debug the target
software. This "Should" requirement may make the emulator a more useful tool
beyond just acting as a fuzz testing harness, and allow the emulator to be used
to investigate bugs identified by fuzz testing. Using the emulator in this way
may give additional insight when answering \refrq{1}.

\begin{table}[H]
    \centering
    \begin{tabular}[c]{|l|p{10cm}|}
        \hline
        \definereq{EMU-1} &
        The emulation software must allow the execution of target software on an x86\_64 development computer
        \\
        \hline
        \definereq{EMU-2} &
        The emulation software must include all functionality required to boot and run the target software in a manner representative of the target hardware
        \\
        \hline
        \definereq{EMU-3} &
        The emulation software must allow data to be input via the hardware peripheral interface of the target software
        \\
        \hline
        \definereq{EMU-4} &
        The emulation software should be able to read data from the hardware peripheral interface of the target software
        \\
        \hline
        \definereq{EMU-5} &
        The emulation software should implement features for aiding debugging and introspection of target software execution
        \\
        \hline
        \definereq{EMU-6} &
        The emulation software must allow the target software to be able to be used as a test harness for a coverage guided fuzzer
        \\
        \hline
        \definereq{EMU-7} &
        The emulation software should allow the target software to be able to be used as a test harness for a black box fuzzer
        \\
        \hline
    \end{tabular}
    \caption{Derived Requirements for the emulator}
    \label{tab:emu-reqs}
\end{table}

%% Protocol Grammar Filter requirements
% the protocol grammar filter shall generate valid input data to the target software
% the protocol grammar filter shall generate known invalid input data to the target software
% the protocol grammar filter shall use a variable length binary stream as input data
% the protocol grammar filter input data generation shall be deterministic

The grammar filter derived requirements are mostly focused towards satisfying \refrq{2}.
% TODO need more here about the pgf filter.
For the filter to be valid for use with a fuzzer, the output of the filter must
be ensured to always be the same given the same input (\refreq{PGF-4}). If this
were not the case, the algorithms in the fuzzer for mutating the input data
would not work correctly. Additionally, any test cases identified by the fuzzer
would not be able to be replayed to investigate any bugs found.

Coverage guided fuzzers typically provide input in the form of a stream of
bytes, and so the protocol grammar filter also needs to be able to handle
streams of bytes of differrent sizes (\refreq{PGF-3}).

\begin{table}[H]
    \centering
    \begin{tabular}[c]{|l|p{10cm}|}
        \hline
        \definereq{PGF-1} &
        The protocol grammer filter must generate valid input data to the target software
        \\
        \hline
        \definereq{PGF-2} &
        The protocol grammer filter must generate known invalid input data to the target software
        \\
        \hline
        \definereq{PGF-3} &
        The protocol grammer filter must use a variable length stream of bytes as input data
        \\
        \hline
        \definereq{PGF-4} &
        The protocol grammer filter must generate input data in a deterministic manner
        \\
        \hline
    \end{tabular}
    \caption{Derived Requirements for the protocol grammar filter}
    \label{tab:pgf-reqs}
\end{table}



%% Software Design?

% [ ] Outline software design needed to meet requirements
% - list all software developed, why they are needed, what requirement they satisfy
% - explain design decisions, such as STM32, DBC, minimal rtos, unicorn
% [ ] Outline tasks needed to meet requirements

% Software tools required:
%  * ground software:
%   ** send and receive spacepackets
%   ** blackbox fuzzer
%    *** grammar filter
%    *** raw data
%  * rehosting software (emulator)
%   ** interrupt handling

\section{Design}


\subsection{Target Hardware}
The first design decision that had to be made, building on \refreq{FSW-1}, was
to select a microcontroller on which to build the flight software. This was
especially critical, not just to ensure the flight software was representative,
but as the hardware selected would have a large impact on the design of the
emulator and fuzzer.

\citet{Cratere_2024} outlines alot of useful information about the state of the
art cubesat OBC technology, and this information was used as a reference when
designing the flight software component of the work. They identify Arm Cortex-M
microcontrollers as the most popular option. They also compose a list of flight
hardware analysed, identifying the operating system and processor on each.
From this comprehensive list, many different cubesat OBC's can be seen to use
STM32 devices, such as EnduroSat \citep{EnduroSat_OBC} and NanoAvionics
\citep{Nano_OBC}. Many other OBC's, such as the AAC-ClydeSpace Kryten
\citep{Clydespace_Kryten}, use System On Chip (SoC) processors, where an FPGA
and ARM microcontroller are integrated with memory into the same package. The
Kryten uses a Microsemi SmartFusion 2 SoC, which includes an ARM Cortex-M3
microcontroller.

Real flight hardware, such as those mentioned above, is expensive. The
EnduroSat OBC is listed as 4,300 euros \citep{EnduroSat_OBC}. Therefore, for
this work an STM32F411 device was chosen. The STM32F411 is a ARM Cortex-M4, and
is available as a consumer development board called BlackPill, which was used
as the target hardware throughout the work \citep{blackpill_info}. The
Cortex-M4 architechture is similar to Cortex-M3, like as used in the Kryten
\citep{Clydespace_Kryten}, except that Cortex-M4 may includes support for
hardware floating point arithmetic and DSP instructions \citep{Cortex_M3}
\citep{Cortex_M4}. The STM32 chip used in the EnduroSat OBC is a Cortex-M7
\citep{EnduroSat_OBC}, which is very similar to a Cortex-M4, but with better
performance and power efficiency, and a larger 64-bit instruction and data bus
\citep{Cortex_M7}.

Additionally, STM32F411 microcontrollers have many peripherals and interfaces
which are common in cubesat OBC boards, such as Inter-Integrated Circuit (I2C),
Serial Peripheral Interface (SPI), and Universal Synchronous and Asynchronous
Receiver Transmitter (USART) \citep{Cratere_2024}. Therefore, the STM32F411
blackpill development board is sufficiently representative of OBC hardware used
in the New Space industry for the purposes of this work.

Typically, a CubeSat OBC communicates with a ground station using a S-band
radio (known as a TMTC Radio). The radio is usually connected to the OBC using
a communication bus such as CAN, Serial or SpaceWire. An example TMTC radio is
the Satlab SRS-3, which provides both CAN and RS-422 interfaces
\citep{Satlab_SRS3}. Therefore, a serial communication interface (between the
OBC and Radio) will be representative. The system boundary for emulating and
fuzzing the flight software is at the OBC peripherals, and so for this work any
processing carried out by the TMTC radio, such as foward error correction, does
not need to be considered in the design.

\subsection{Flight Software}

% - Written in C? The only real choice
% - Bare Metal
%   - HAL
%   - Startup Code
%   - Design by Contract
% - Actions, Parameters, Telemetries, RPC
% - Spacepacket Interface
% - Bare Metal RTOS inspired by MIROS (rather than FreeRTOS or RTEMS, easier to reason about and emulate for the demonstration of the project)
% - CMSIS? Avoid using relying on CMSIS to reduce the amount of perhipherals and mmio that would need to be emulated.

\subsection{Emulator}

% - Python
% - Unicorn
%   - Integration with AFL for coverage guided fuzzing
%   - Options include QEMU, Unicorn is QEMU, cite papers stating it as a form of emulation?
% - Peripheral Models
%   - use unicorn hooks to capture memory accesses to specific mmio addresses, and store those in python models

% Design Rationale:
% - why unicorn / python for peripheral modelling

\end{document}
