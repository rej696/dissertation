\documentclass[../report.tex]{subfiles}
\graphicspath{{\subfix{../images}}}
\begin{document}

\section{Requirements}
In order to answer the research questions outlined in
\textbf{\autoref{sec:rq}}, a set of requirements needed to identified to drive
the design of the software that needed to be written.

% Requirements and Design
\subsection{User Requirements} \label{sec:user-req}

\textbf{\autoref{tab:rq1-reqs}} shows user requirements derived to satisfy
\refrq{1}. This requirement set shows the developments fundamental to the
project, namely the development of flight software (\refreq{RQ1-1}), emulator
(\refreq{RQ1-2}) and coverage guided fuzzer (\refreq{RQ1-3}), as "Must"
requirements. The development of a blackbox fuzzer was deemed to be a "Should"
requirement (\refreq{RQ1-6}, \refreq{RQ1-7}). Running fuzz tests on the
hardware (\refreq{RQ1-5} and \refreq{RQ1-8}) were identified as "Could"
requirements, as potential extensions to the work.

\begin{table}[H]
    \centering
    \begin{tabular}[c]{|l|p{10cm}|}
        \hline
        \definereq{RQ1-1} &
        A target software program must be developed that is representative of space flight command and data handling software
        \\
        \hline
        \definereq{RQ1-2} &
        An emulator must be developed that is able to execute the target software on an x86\_64 Linux computer
        \\
        \hline
        \definereq{RQ1-3} &
        A coverage guided fuzzer must be implemented to fuzz test the target software
        \\
        \hline
        \definereq{RQ1-4} &
        The coverage guided fuzzer must be able to fuzz the target software running on the emulator
        \\
        \hline
        \definereq{RQ1-5} &
        The coverage guided fuzzer could be able to fuzz the target software running on the target hardware
        \\
        \hline
        \definereq{RQ1-6} &
        A blackbox fuzzer should be implemented to fuzz test the communication interface of the target software
        \\
        \hline
        \definereq{RQ1-7} &
        The blackbox fuzzer should be able to fuzz the target software running on the emulator
        \\
        \hline
        \definereq{RQ1-8} &
        The blackbox fuzzer could be able to fuzz the target software running on the target hardware
        \\
        % \hline
        % \definereq{A} & Should &
        % \\
        % \hline
        % \definereq{B} & \textit{Could} &
        % \\
        % \hline
        % \definereq{C} & \textbf{Must} &
        % \\
        \hline
    \end{tabular}
    \caption{User Requirements to satisfy Research Question 1}
    \label{tab:rq1-reqs}
\end{table}

\textbf{\autoref{tab:rq2-reqs}} shows user requirements derived to satisfy \refrq{2}.
This requirement set has some overlap with those identified to satisfy
\refrq{1} in \textbf{\autoref{tab:rq1-reqs}}. Both require some target software
to be developed (\refreq{RQ1-1}, \refreq{RQ2-1}) and both require the
implementation of a coverage guided fuzzer (\refreq{RQ1-3}, \refreq{RQ2-2}).

\begin{table}[H]
    \centering
    \begin{tabular}[c]{|l|p{10cm}|}
        \hline
        \definereq{RQ2-1} &
        A target software program must be developed that implements a communication protocol
        \\
        \hline
        \definereq{RQ2-2} &
        The communication interface of the target software must be representative of a space flight command and data handling system
        \\
        \hline
        \definereq{RQ2-3} &
        A coverage guided fuzzer must be implemented to fuzz test the communication interface of the target software
        \\
        \hline
        \definereq{RQ2-4} &
        A protocol grammar filter must be developed that ensures a random bytestream satisfies the grammar of the protocol
        \\
        \hline
        \definereq{RQ2-5} &
        The protocol grammar filter must be able to be integrated with the coverage guided fuzzer
        \\
        \hline
    \end{tabular}
    \caption{User Requirements to satisfy Research Question 2}
    \label{tab:rq2-reqs}
\end{table}

% TODO how will both these requirements sets satisfy the research questions?

\subsection{Derived Requirements} \label{sec:derived-req}

The user requirements outlined to satisfy the research questions in
\autoref{sec:user-req} identified three main pieces of software that needed to
be developed. A representative flight software (\refreq{RQ1-1},
\refreq{RQ2-2}), some emulation software capable of being used as a test
harness for a coverage guided fuzzer (\refreq{RQ1-2}, \refreq{RQ2-3}), and some
software for generating some valid communication according to a protocol
grammar (\refreq{RQ2-4}).

Additional requirements for each of these programs, namely the Flight Software
(FSW), Emulation (EMU) and Protocol Grammar Filter (PGF), were derived from the
user requirements and are outlined and discussed below.

%% TODO how to show what was derived from where? a seperate table after defining the requirements?

%% FSW Requirements
% Target software must be implemented to target arm cortex m
% Target software must implement the CCSDS spacepacket protocol as a communication interface
% The communication interface should include a packet verification method, such as a checksum or CRC
% The target software must allow the remote triggering of procedures
% The target software must allow the retrieval of telemetry data
% The target software could allow the setting of configuration data
% The target software could allow the retrieval of configuration data
% There must be a method to communicate with the target software from an x86\_64 development computer
% The target software should implement functionality similar to a real time operating system
%% TODO critique these requirements and provide rationale, showing how they will satisfy the research question

\begin{table}[H]
    \centering
    \begin{tabular}[c]{|l|p{10cm}|}
        \hline
        \definereq{FSW-1} &
        The target software must be implemented to target an ARM Cortex-M microcontroller
        \\
        \hline
        \definereq{FSW-2} &
        The target software must be implement the spacepacket protocol over a hardware communication interface
        \\
        \hline
        \definereq{FSW-3} &
        The target software should include a method of verifying packets received over the communication interface, such as a checksum
        \\
        \hline
        \definereq{FSW-4} &
        The target software must allow the remote triggering of procedures
        \\
        \hline
        \definereq{FSW-5} &
        The target software must allow the retrieval of telemetry data
        \\
        \hline
        \definereq{FSW-6} &
        The target software could allow the setting of configuration data
        \\
        \hline
        \definereq{FSW-7} &
        The target software could allow the retrieval of configuration data
        \\
        \hline
        \definereq{FSW-8} &
        There must be a method to communicate with the target software from an x86\_64 development computer
        \\
        \hline
        \definereq{FSW-9} &
        The target software should implement functionality similar to a real time operating system
        \\
        \hline
    \end{tabular}
    \caption{Derived Requirements for the target flight software}
    \label{tab:fsw-reqs}
\end{table}

\autoref{tab:fsw-reqs} shows the derived flight software requirements.
\refrq{2} is focused on fuzz testing the communication interface with the
flight software, and so several of these requirements clarify that interface.
The use of spacepacket protocol (\refreq{FSW-2}), the implementation of Real
Time Operating System (RTOS) functionality (\refreq{FSW-9}), and the need for
remote procedure triggering (\refreq{FSW-4}), are all identified to ensure the
FSW follows standard FSW design decisions (outlined in \autoref{chap:lit-rev}),
as required by \refreq{RQ1-1}.

Other requirements, such as \refreq{FSW-8}, are included to ensure that the
design of the FSW is easy to fuzz test, and thus satisfy both \refrq{1} and
\refrq{2}.

\refreq{FSW-1} specifies that the target software be designed to run on an ARM
Cortex-M microcontroller. This microcontroller architecture is very common in
space applications \citep{Cratere_2024}, so also helps to satisfy
\refreq{RQ1-1}. Additionally, this is the architecture supported by state of
the art rehosting fuzzers like Hoedur and Fuzzware \citep{Hoedur_2023,
Fuzzware_2022}.

\autoref{tab:emu-reqs} outlines derived requirements for the emulator. Some of
these requirements are directly derived from the User Requirements, such as
\refreq{EMU-1} and \refreq{RQ1-2}. Others are fundamental for the emulator to
be able to be used to answer \refrq{2}, such as \refreq{EMU-6}.

\refreq{EMU-5} aims to enable the emulator to be used to debug the target
software. This "Should" requirement may make the emulator a more useful tool
beyond just acting as a fuzz testing harness, and allow the emulator to be used
to investigate bugs identified by fuzz testing. Using the emulator in this way
may give additional insight when answering \refrq{1}.

\begin{table}[H]
    \centering
    \begin{tabular}[c]{|l|p{10cm}|}
        \hline
        \definereq{EMU-1} &
        The emulation software must allow the execution of target software on an x86\_64 development computer
        \\
        \hline
        \definereq{EMU-2} &
        The emulation software must include all functionality required to boot and run the target software in a manner representative of the target hardware
        \\
        \hline
        \definereq{EMU-3} &
        The emulation software must allow data to be input via the hardware peripheral interface of the target software
        \\
        \hline
        \definereq{EMU-4} &
        The emulation software should be able to read data from the hardware peripheral interface of the target software
        \\
        \hline
        \definereq{EMU-5} &
        The emulation software should implement features for aiding debugging and introspection of target software execution
        \\
        \hline
        \definereq{EMU-6} &
        The emulation software must allow the target software to be able to be used as a test harness for a coverage guided fuzzer
        \\
        \hline
        \definereq{EMU-7} &
        The emulation software should allow the target software to be able to be used as a test harness for a black box fuzzer
        \\
        \hline
    \end{tabular}
    \caption{Derived Requirements for the emulator}
    \label{tab:emu-reqs}
\end{table}

%% Protocol Grammar Filter requirements
% the protocol grammar filter shall generate valid input data to the target software
% the protocol grammar filter shall generate known invalid input data to the target software
% the protocol grammar filter shall use a variable length binary stream as input data
% the protocol grammar filter input data generation shall be deterministic

The grammar filter derived requirements are mostly focused towards satisfying \refrq{2}.
% TODO need more here about the pgf filter.
For the filter to be valid for use with a fuzzer, the output of the filter must
be ensured to always be the same given the same input (\refreq{PGF-4}). If this
were not the case, the algorithms in the fuzzer for mutating the input data
would not work correctly. Additionally, any test cases identified by the fuzzer
would not be able to be replayed to investigate any bugs found.

State of the art rehosting based fuzzers like Hoedur and Fuzzware provide input
in the form of a stream of bytes \citep{Hoedur_2023, Fuzzware_2022}, and so the
protocol grammar filter also needs to be able to handle streams of bytes of
different sizes (\refreq{PGF-3}).

\begin{table}[H]
    \centering
    \begin{tabular}[c]{|l|p{10cm}|}
        \hline
        \definereq{PGF-1} &
        The protocol grammar filter must generate valid input data to the target software
        \\
        \hline
        \definereq{PGF-2} &
        The protocol grammar filter must generate known invalid input data to the target software
        \\
        \hline
        \definereq{PGF-3} &
        The protocol grammar filter must use a variable length stream of bytes as input data
        \\
        \hline
        \definereq{PGF-4} &
        The protocol grammar filter must generate input data in a deterministic manner
        \\
        \hline
    \end{tabular}
    \caption{Derived Requirements for the protocol grammar filter}
    \label{tab:pgf-reqs}
\end{table}

\section{Design}

\subsection{Target Hardware}
The first design decision that had to be made, building on \refreq{FSW-1}, was
to select a microcontroller on which to build the flight software. This was
especially critical, not just to ensure the flight software was representative,
but as the hardware selected would have a large impact on the design of the
emulator and fuzzer.

\citet{Cratere_2024} outlines useful information about the state of the
art CubeSat OBC technology, and this information was used as a reference when
designing the flight software component of the work. They identify Arm Cortex-M
microcontrollers as the most popular option. They also compose a list of flight
hardware analysed, identifying the operating system and processor on each.
From this comprehensive list, many different CubeSat OBCs can be seen to use
STM32 devices, such as EnduroSat \citep{EnduroSat_OBC} and NanoAvionics
\citep{Nano_OBC}. Many other OBCs, such as the AAC-ClydeSpace Kryten
\citep{Clydespace_Kryten}, use System On Chip (SoC) processors, where an FPGA
and ARM microcontroller are integrated with memory into the same package. The
Kryten uses a Microsemi SmartFusion 2 SoC, which includes an ARM Cortex-M3
microcontroller.

Real flight hardware, such as those mentioned above, is expensive. The
EnduroSat OBC is listed as 4,300 euros \citep{EnduroSat_OBC}. Therefore, for
this work an STM32F411xE device was chosen. The STM32F411xE is a ARM Cortex-M4, and
is available as a consumer development board called blackpill, which was used
as the target hardware throughout the work \citep{blackpill_info}. The
Cortex-M4 architecture is similar to Cortex-M3, like as used in the Kryten
\citep{Clydespace_Kryten}, except that Cortex-M4 may includes support for
hardware floating point arithmetic and DSP instructions \citep{Cortex_M3}
\citep{Cortex_M4}. The STM32 chip used in the EnduroSat OBC is a Cortex-M7
\citep{EnduroSat_OBC}, which is very similar to a Cortex-M4, but with better
performance and power efficiency, and a larger 64-bit instruction and data bus
\citep{Cortex_M7}.

Furthermore, STM32F411xE microcontrollers have many peripherals and interfaces
which are common in CubeSat OBC boards, such as Inter-Integrated Circuit (I2C),
Serial Peripheral Interface (SPI), and Universal Synchronous and Asynchronous
Receiver Transmitter (USART) \citep{Cratere_2024}. Additionally STM32F4
microcontrollers were used in the case study conducted by
\citet{Scharnowski_2023}. Therefore, the STM32F411xE blackpill development board
is sufficiently representative of OBC hardware used in the New Space industry
for the purposes of this work.

Typically, a CubeSat OBC communicates with a ground station using an S-band
radio (known as a TMTC Radio). The radio is usually connected to the OBC using
a communication bus such as CAN, Serial or SpaceWire. An example TMTC radio is
the Satlab SRS-3, which provides both CAN and RS-422 interfaces
\citep{Satlab_SRS3}. Therefore, a serial communication interface (between the
OBC and Radio) will be representative. The system boundary for emulating and
fuzzing the flight software is at the OBC peripherals, and so for this work any
processing carried out by the TMTC radio, such as forward error correction, does
not need to be considered in the design.

\subsection{Flight Software} \label{sec:fsw-design}

As outlined in the User Requirements (\autoref{sec:user-req}) the FSW component
of the work needed to provide enough features to successfully answer the
research questions. In the derived FSW requirements (\autoref{tab:fsw-reqs})
the required features of the FSW are outlined, such as a spacepacket interface
and use of an RTOS. However, there were many design choices not constrained
by these requirements. It was anticipated that developing the emulator and
running the fuzzer would be the most difficult tasks. Therefore, in order to
make the emulation and fuzz testing as easy as possible, most of the FSW design
and implementation decisions were made with transparency and simplicity in
mind.

Embedded software typically uses a layered architecture, with libraries
providing the application software with hardware abstractions and real time
operating system primitives \citep{Cratere_2024}. Therefore, the FSW design
included layers for the Hardware Abstraction Layer (HAL), RTOS, and application
code.

The HAL would handle interaction with the USART peripheral in the STM32
blackpill, providing functions to read and write data. The HAL would need to be
able to handle interrupts from the USART peripheral. The USART peripheral would
be the spacepacket interface, meaning spacepackets could be sent to the
blackpill from a development machine using an FTDI cable, satisfying
\refreq{FSW-8}.

The application was designed to read spacepacket telecommands from the USART
peripheral using the HAL. Then parse each spacepacket, rejecting any with
invalid headers or checksums, and parse and execute the commands they
contained. The application would include implementations for handling each
different class of command required by the FSW User Requirements
(\autoref{tab:fsw-reqs}). The application process identifier (APID) field of
the spacepacket header \citep{Ccsds_spp} was used to select which type of
command was included in the spacepacket data field.

The application included multiple RTOS Threads. One thread would read
data received over the spacepacket USART interface, another would handle
parsing received spacepackets and executing the commands. A third thread would
be used to provide a liveness check, periodically printing to a debug USART
peripheral.

The FSW also needed to include code for handling the boot process of the
program. This included initialising the HAL and RTOS, and registering any
command handlers for the application.

\subsection{Emulator}

% - Python
% - Unicorn
%   - Integration with AFL for coverage guided fuzzing
%   - Options include QEMU, Unicorn is QEMU, cite papers stating it as a form of emulation?
% - Peripheral Models
%   - use unicorn hooks to capture memory accesses to specific mmio addresses, and store those in python models

% Design Rationale:
% - why unicorn / python for peripheral modelling

The design of the emulator was driven by the need to integrate it with a
coverage guided fuzzer (\refreq{EMU-6}). As investigated in the literature
review (\autoref{sec:lit-rev:fuzz}), QEMU and Unicorn are commonly used to
implement emulators. \citet{Yun_2022} found that QEMU was the most popular
emulator, while Unicorn was lighter weight and more flexible. Furthermore,
current state of the art rehosting fuzzers, like Fuzzware and Hoedur, make use
of Unicorn \citep{Fuzzware_2022, Hoedur_2023}. In addition, the popular fuzzer
AFL++ has a mode for fuzz testing Unicorn emulation harnesses
\citep{UnicornMode}. Therefore, it was decided to build the emulator using
Unicorn, and use AFL++ as the coverage guided fuzzer, satisfying
\refreq{EMU-1} and \refreq{EMU-6}.

Unicorn provides an API for loading binary firmware into a memory map and
executing it. It also provides abstractions for attaching function hooks to
memory accesses and instruction fetches \citep{Unicorn}. The emulator software
developed would wrap the unicorn API and implement hooks for managing the
correct execution of the FSW image.

The emulator included models of hardware peripherals used by the FSW, such as
the USART peripheral used as the spacepacket interface. These would use MMIO
hooks to manage the state and behaviours of the peripherals, such as triggering
interrupts and setting bits in MMIO registers as necessary. The USART
peripheral model would need to be able to accept input data (from the fuzzer,
or PGF) and "send" it to the FSW (\refreq{EMU-3}, \refreq{EMU-4}). The full set
of peripheral models and features needed for the correct execution of the FSW
was determined iteratively as the FSW was developed alongside the emulator.

The emulator was designed to be able to operate both under the fuzzer, and
separately with manually written input data, to allow for testing and
debugging, by running it from the command line with different flags
(\refreq{EMU-5}). This required the development of a command line wrapper
application for the emulator.

\subsection{Protocol Grammar Filter}

To satisfy \refreq{PGF-1} the design of the FSW must be considered.
\autoref{sec:fsw-design}, \refreq{FSW-2} and \refreq{FSW-3} indicate that the
FSW will implement the spacepacket protocol with a checksum field. Therefore,
the PGF was designed to create a lazy stream of semantically valid input data
from an unknown input stream. The PGF was designed to be able to be used in a
data processing pipeline, with the aim of being able to accept input and route
output to and from different sources. For example, receiving input data from a
blackbox fuzzer and routing output to the target hardware, or alternatively
receiving input from a coverage guided fuzzer and routing it to an emulated
peripheral.

In order to provide a deterministic input interface for the coverage guided
fuzzer, the PGF had to assign meaning to each of the bytes in the input,
effectively determining its own input protocol. The PGF needed to be able to
inject known errors \refreq{PGF-2}, and so needed to be able to determine which
error condition should be injected from the input data.

The rationale for designing and using the PGF, apart from to improve the
effectiveness of a fuzzer when testing a communication interface, was to
improve the impact of each bit of information generated by the fuzzer, and so
reduce the input overhead \citep{Fuzzware_2022, Hoedur_2023,
Fuzztruction_2023}. Therefore, the protocol structure for the input to the PGF
needed to ensure that all bits used would alter the PGF output in a meaningful
way. The spacepacket protocol defines several header fields that use multiple
bits, but are either compile time configured or only have one correct value,
such as the version field \citep{Ccsds_spp}. Therefore, the filter needed to
use constant values for those fields in the spacepacket header, and only use a
single bit to switch between correct and incorrect version fields.

Some spacepacket fields, such as the application process identifier (APID) and
packet data field, needed their values to be modified by the fuzzer. In the
case of the APID, there were only four valid APID values for the FSW
application. Therefore, three bits would be required to cover all valid values
and the invalid APID case. Given the APID field is 11 bits long
\citep{Ccsds_spp}, this optimisation to the input protocol would greatly reduce
the input overhead.

\subsection{Ground Software}

In order to test and run the FSW on the target hardware, a ground software
application would need to be developed to communicate with the flight software.
This application would be run on the development machine to send valid
spacepackets over the serial FTDI cable to the STM32 blackpill USART interface.

The ground software was designed to be a command line tool, like the emulator,
with command line arguments and options to send different spacepacket
telecommands and use different features. The ground software needed to generate
valid input data according to the implementation of the communication interface
in the FSW. The ground software could achieve this by manually writing packet
generation code, or integrating the Protocol Grammar Filter.

In order to satisfy \refreq{RQ1-8}, this command line tool was designed to be
able to be integrated with a blackbox fuzzer. Therefore, the ground software
would need to be able to monitor the liveness check from the FSW, and report
any crashes.

\end{document}
