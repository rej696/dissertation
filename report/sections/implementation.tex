\documentclass[../report.tex]{subfiles}
\graphicspath{{\subfix{../images}}}
\begin{document}

%% Software Design?

% [ ] Outline software design needed to meet requirements
% - list all software developed, why they are needed, what requirement they satisfy
% - explain design decisions, such as STM32, DBC, minimal rtos, unicorn
% [ ] Outline tasks needed to meet requirements

% Software tools required:
%  * ground software:
%   ** send and receive spacepackets
%   ** blackbox fuzzer
%    *** grammar filter
%    *** raw data
%  * rehosting software (emulator)
%   ** interrupt handling

All the software implemented is included in a single repository, using a
makefile to build, program, and run the FSW on the target hardware and the
emulator. Files for the flight software implementation are included in the
\lstinline|fsw| directory. Files for the emulator implementation are included
in the \lstinline|emu| directory, and files for the protocol grammar filter
are included in the \lstinline|pgf| directory. The ground software python tool
is located at the project root and called \lstinline|gsw|. The \lstinline|pgf| is
not a stand-alone application, like the \lstinline|emu| and \lstinline|gsw|
tools, but a library containing code that allows the protocol grammar filter to
be used in both the emulator and the ground software.

% Implementation (and Testing?)
% FIXME does implementation section include discussion of implementation decisions? Yes probably.
\section{Flight Software}

Key features of the FSW implementation are outlined and discussed in the
following sections.

\subsection{C Programming Language} %TODO is this necssary as a sub section?

The most logical and popular choice of programming language for embedded flight
software is the C programming language. C is used in FSW frameworks like Gen1
and cFS \citep{Bal_FSDK, Nasa_cFS}. Languages like C++ and Ada are also
commonly used: C++ is used in the F Prime FSW framework \citep{Nasa_fprime}.
Recently, modern languages like Micropython, Rust and Zig have started to
become popular for embedded devices. Rust was used by the University of
Stuttgart for their sat-rs flight software library, which flew as an experiment
on the ESA OPS-SAT satellite \citep{satrs, opsat}. However, support for these
modern languages is still in progress, and currently do not have the stability
required for commercial projects. In preparing for developing the flight
software, an initial bare metal program for the STM32 blackpill was built using
Zig, but it was decided that using Zig over C would add unnecessary complexity
and reduce the representativeness of the FSW.


\subsection{Hardware Abstraction Layer} \label{sec:fsw-hal}

ARM Cortex-M devices have two distinct layers of hardware implementation. The
core hardware architecture, what makes a chip an ARM Cortex-M, is provided by
ARM, and includes the basic designs needed for the device to function. This
layer includes the processor, and hardware peripherals like the nested vector
interrupt controller (NVIC), which manages and dispatches interrupts; or the
serial wire debug (SWD) interface, which provides developers with a universal
interface for programming and debugging firmware on ARM Cortex-M
microcontrollers \citep{armcm4_manual}. ARM provide a software library for
interacting with this layer of the hardware, called the Common Microcontroller
Software Interface Standard (CMSIS) \citep{CMSIS}. CMSIS provides a level of portability to
embedded software by separating the interface for interacting with hardware
from the implementation, similar to a vendor HAL.

An ARM Cortex-M chip will include another layer of hardware implementation,
designed by the vendor (STMicroelectronics for the STM32F411xE). This
layer includes software interfaces for additional hardware peripherals, such as
memory, IO devices like USART or I2C, clocks, GPIOs and so on. Vendors, like
STMicroelectronics, typically provide a HAL, which is a software library for
interfacing with the vendor hardware \citep{stm32hal}. These HALs can vary in
size and complexity from vendor to vendor, and many are either built on top of
CMSIS or compatible with CMSIS.
% TODO CUT?
Often these HAL implementations are part of a developer tooling
ecosystem, and are integrated into a proprietary IDE. While this can make
getting started with development easier, it can result in vendor lock in, where
skills and knowledge gained using one vendor are difficult to transfer to
another vendor. Additionally, it can be difficult to introspect the binary
produced through these IDEs, and therefore not easy to determine exactly what
code and functions have been included.
% END CUT
It was determined that only elements of the hardware abstraction layer required
for the FSW should be included to reduce the complexity of the emulator
development. Therefore, a minimal HAL (\lstinline|fsw/{inc,src}/hal|) was
developed by consulting the STM32F411xC/E Reference Manual and Datasheet
\citep{stm32f4_manual, stm32f4_datasheet}, and the ARM Cortex-M4 User Manual
\citep{armcm4_manual}. This decision also led to the manual implementation of
several parts of the FSW project that are often provided by the vendor. Namely
the start-up code (\lstinline|fsw/src/hal/startup.c|), a linkerscript for the
blackpill (\lstinline|fsw/stm32f411xx.ld|), and a \lstinline|make| build system
for compiling the project with the \lstinline|arm-none-eabi-gcc| toolchain
(\lstinline|Makefile|). The first task in implementing the FSW was to implement
these files.

\subsubsection{Linker Script and Start-up Code} \label{sec:linkerscript}

The linkerscript maps the software linker sections into the memory locations
for the STM32F411xE. Section 5 of the STM32F411xC/E datasheet outlines the
address ranges for flash and sram memory \citep{stm32f4_datasheet}. There are
several notable sections in the linker script. The \lstinline|.vectors| section
holds the vector table for the firmware, which must be located at the beginning
of the flash memory. The \lstinline|.vectors| section is implemented as an
array of function pointers in the \lstinline|startup.c| file, as defined in
section 2.3.4 in the Cortex-M4 User Manual \citep{armcm4_manual}. The
\lstinline|.text| section includes the machine code compiled from the source
code, and the \lstinline|.rodata| section includes any data the compiler has
determined as non writeable. These sections are written to the flash memory,
which only has read/execute permissions.

Traditionally, the stack is defined at the end of the working memory, which in
the case of the blackpill is sram. However, as \citet{miro_stack} outlines,
this decision can lead to a stack overflow overwriting data in other memory
sections, and potentially going undetected until a catastrophic failure.
Instead, the stack can be placed at the beginning of the sram. As the stack
grows down towards the beginning of sram, any stack overflow will result in an
invalid memory access and trigger a hardware exception, rather than corrupting
other data in memory \citep{miro_stack}. This technique is especially
beneficial when considering that a fuzzer will now be able to immediately
detect a stack overflow if one is triggered.

The \lstinline|.data| section includes initialised global data. The initial
values for these variables must be stored in non-volatile memory (flash).
However, as the data is mutable, it must be loaded into sram and data accesses
must reference addresses in that section. The linker provides a mechanism to
set the load address for a section. The \lstinline|.bss| section is for
uninitialised global data, and so only needs to be included in sram. However,
it is good practice to initialise the memory in the \lstinline|.bss| section on
boot, to reduce the risk associated with using uninitialised variables.

As mentioned, the start-up code included a definition of the vector table, which
includes functions for interrupt handlers and arm exception handlers.
The \lstinline|startup.c| file uses symbols defined in the linkerscript to
place a pointer to the end of the stack at the beginning of the vector table.
A \lstinline|Reset_Handler| function is also defined, which loads the
\lstinline|.data| section into sram, initialises the \lstinline|.bss| section
to zero, and calls the \lstinline|main| function.

The last step required to having a minimum runnable firmware was compiling the
code for the target hardware. A \lstinline|Makefile| file was written which
included commands for compiling object files and linking them into an elf using
\lstinline|arm-none-eabi-gcc|, and extracting the binary code to be loaded into
the flash memory from the elf using \lstinline|arm-none-eabi-objcopy|. Notable
compiler flags included telling the compiler to optimise for size
(\lstinline|-Os|) to ensure the binary would fit into the limited flash memory
available, as well as enabling as many compiler warnings as possible to catch
errors at compile time.

After implementing all the above, a minimum firmware image could be compiled
and loaded onto the target hardware.

\subsubsection{Peripheral Drivers}

The HAL needed additional functionality to allow the firmware to interact with
hardware peripherals, in order to satisfy the requirement of communicating over
a hardware interface \refreq{FSW-2}.

The first peripheral implemented was for controlling the general purpose input
output (GPIO) pins of the blackpill. Functions were implemented for setting the
mode and alternate function of a GPIO pin, as well as writing a value to that
pin. Toggling the state of a GPIO is a useful method for debugging embedded
software, and setting alternate GPIO functions was required to be able to use the
USART peripherals. The procedure for configuring a GPIO pin is outlined in the
STM32F411xC/E Reference Manual \citep{stm32f4_manual}. The GPIO hardware in the
STM32 can be enabled by setting the appropriate values in the
\lstinline|RCC_AHB1ENR| register on the Reset and Clock Control (RCC)
peripheral \citep{stm32f4_manual}. A structure for accessing the RCC peripheral
registers was included in \lstinline|fsw/inc/rcc.h|.

% - SysTick
The SysTick peripheral is a standard component of ARM Cortex-M devices, which
provides a mechanism to trigger an interrupt when the hardware system timer
counts down to zero. This peripheral is commonly used as an operating system
tick, and to implement software timers \citep{armcm4_manual}. The SysTick
peripheral exposes several registers to configure the device, notably the
\lstinline|LOAD| register that sets the number of clock ticks between interrupt
triggers. Given the clock frequency of the blackpill was 16MHz, the SysTick
peripheral was configured to fire a SysTick interrupt every millisecond. The
implementation in \lstinline|fsw/src/hal/systick.c| was crucial to
the functioning of the RTOS. The driver also implemented a software timer that
allowed for periodic behaviour in the system. This was needed in order to
implement a liveness check in the FSW.

% - Uart
Finally, a driver was implemented for interacting with the USART peripheral
(\lstinline|fsw/src/hal/uart.c|). The first iteration of the USART
driver implemented synchronous blocking reading and writing from the
peripheral. The driver used previously implemented HAL functions, such as
\lstinline|gpio_set_mode|, to enable the USART peripheral and setup the
associated GPIO pins correctly.
% TODO Move?
% Later in the project, after implementing parts
% of the application layer and RTOS, the ability of the FSW to receive
% spacepacket data over the USART was tested on target. At this point, using the
% synchronous uart implementation caused packet data to be dropped if the data was
% received while a different thread was executing.
% ENDMOVE?
Asynchronous reading of data from the USART was implemented using interrupts.
Interrupt Service Routines (ISR) for each USART peripheral were implemented,
which overrode the weakly defined aliases in the vector table. A circular
buffer data structure was created at \lstinline|fsw/src/utils/cbuf.c|, which
allowed data to be continuously added for consumption at a later point. Each
USART peripheral was mapped to a distinct \lstinline|cbuf| object. The ISR read
a byte from the data register and appended it to the circular buffer. The
USART driver then provided an interface for reading the data stored in the
circular buffer.
% TODO Move?
% This implementation resulted in no data being dropped by
% reading the USART.
% ENDMOVE

It was initially decided to exclude CMSIS from the project in addition to the
vendor HAL. However, when the USART HAL implementation added non blocking
(interrupt driven) reading of USART data, further interfacing with the NVIC
peripheral was required. It was determined that re-implementing the NVIC API in
CMSIS would make the project more complex than including the CMSIS library.
Therefore a subset of files necessary for using an ARM Cortex-M4 were included
from CMSIS-core \citep{CMSIS} in \lstinline|fsw/CMSIS|, and were configured for
the STM32F411xE (\lstinline|fsw/inc/hal/stm32f4_blackpill.h|).

In retrospect, while implementing a reduced HAL did make it simpler to develop
the emulator in parallel to the flight software, CMSIS should have been
included from the start of the project. This would have avoided a lot of
difficulty when developing the lower level parts of the FSW.

\subsection{Design By Contract}

The FSW made use of design by contract (DBC) as a methodology for ensuring the
correctness of the implementation. The \lstinline|dbc_assert.h| header file
from \citet{dbc_assert} was integrated into the FSW. This header file provided
macros for DBC primitives like pre and post conditions and assertions, which
were used throughout the software to check for common C errors, such as passing
null pointers to functions and size parameters exceeding static limits. When a
DBC assertion check fails, such as when a pointer argument is null, the macro
calls a \lstinline|DBC_fault_handler| function. The implementation of this
handler can be tailored according to the application. In the FSW
implementation, the handler prints a DBC failure message and enters an infinite
loop. However, a release version of the FSW might disable the execution of the
fault handler entirely, or have the fault handler automatically trigger a reset
of the system. It was reasoned that DBC would be a complementary approach to
fuzz testing, as the execution of the \lstinline|DBC_fault_handler| function
would imply a bug in the software, and should be easy to detect both on target
and rehosted, by monitoring the debug log and program counter respectively.

\subsection{Real Time Operating System} \label{sec:fsw-rtos}

The implemented RTOS (\lstinline|fsw/src/rtos/thread.c|) provided an
\lstinline|rtos_thread_create| function, which setup an
\lstinline|rtos_thread_t| object. This function took a
\lstinline|rtos_thread_handler_t| as an argument, which is a function pointer
for the thread to be created. This function initialises a separate stack for
the thread according to the Procedure Call Standard for Arm Architecture
(AAPCS) \citep{AAPCS}, which outlines the rules ARM processors and C functions
need to follow when using the stack to call and return from functions. The
\lstinline|rtos_thread_create| function also fills the empty part of the stack
with an easily detectable sequence of 32-bit words, known as stack paint. This
is a feature to aid debugging and optimisation, as the stack paint can act as a
"high water mark", showing the maximum stack usage. Furthermore, reading from
unassigned variables or memory in the stack is more likely to cause a crash
when using stack paint instead of initialising the stack memory to zero, making these
types of errors more detectable during development. Finally, the
\lstinline|rtos_thread_create| function registers the new thread with the RTOS,
indicating it is ready to be run.

The mechanism for context switching between rtos threads uses the PendSV system
interrupt and the \lstinline|rtos_schedule| function. The PendSV interrupt is
set to the lowest interrupt priority, and contains inline assembly code to push
the current register state of the currently executing rtos thread, called the
thread context, onto its stack. The context of the next thread to be executed
is then restored from the stack of the new thread, and execution resumes in the
new thread. The \lstinline|rtos_schedule| function determines which thread to
run next, and if any context switch is required, and triggers the PendSV
interrupt to carry out the context switch. The \lstinline|rtos_schedule|
function is first called by the \lstinline|rtos_run| function to begin
execution of the RTOS, which itself should be called in the main function after
setting up all the threads. The SysTick ISR was modified to call the
\lstinline|rtos_schedule| function periodically, potentially triggering a
context switch every millisecond. The \lstinline|rtos_schedule| function
implemented round-robin scheduling, where all threads are treated equally
without priorities and are executed in cyclic order.

The RTOS implementation also provided functionality for RTOS aware delays,
where if a delay function was called the RTOS would block the thread for the
desired number of ticks, allowing other threads to execute for that period.
This required the implementation of an idle thread that would execute if all
other threads were blocked.

\subsection{Application}

The application code was focused on reading spacepacket data from the USART
peripheral and executing the appropriate application layer handler functions.
As outlined in \autoref{sec:fsw-design}, the FSW used a thread for reading data
from the HAL USART driver, which would be asynchronously read by the packet
handling thread for processing, validation, and command dispatch.

The uart thread and packet thread communicated by reading and writing to
circular buffer, which had a basic mutex implementation, protecting it from
race conditions. This implementation was encapsulated in a C module
(\lstinline|fsw/src/app/frame_buffer.c|), providing thread safe data transfer.

Initially, the FSW handled raw spacepackets. However, after testing on the
hardware and emulator, it quickly became apparent that without any framing,
spacepackets were being dropped or mangled with other data. Therefore, the
spacepackets were encapsulated in KISS frames \citep{kiss}. KISS (Keep It
Simple Stupid) is a common framing layer used on serial interfaces in CubeSats.
For example, libCSP provides an implementation for sending CubeSat Protocol
(CSP) packets encapsulated in KISS frames over a serial interface
\citep{libCSP_kiss}. KISS frames used a special byte value, called FEND, to
indicate the end of a frame. FESC, TFEND and TFESC were other byte values used
to escape a valid data byte that matched the value of FEND or FESC
respectively. Functions for framing and de-framing kiss frames were implemented
in \lstinline|fsw/src/app/kiss_frame.c|. These functions allowed the
packet thread to discretize a continuous stream of input bytes (read from the
\lstinline|frame_buffer|) into distinct spacepackets for processing.

Once the packet thread had collected a distinct spacepacket into a
buffer, it was passed to the \lstinline|spacepacket_process| function in
\lstinline|fsw/src/app/spacepacket.c|. This function parsed and
verified the spacepacket header, and validated the checksum byte at the end of
the packet. If the validation was successful, the APID extracted from the
packet header was used to index a map of application layer handlers, and the
appropriate handler would be called with the spacepacket data payload.

After the application layer handler returned, a telemetry (TM) spacepacket was
built, wrapping the exit code and any returned data from the application
handler. The TM packet was then returned to the packet thread, which
encapsulated it in a KISS frame, and sent it synchronously to the USART device.

As discussed in \autoref{sec:fsw-design}, the application layer consisted of
Action, Parameter and Telemetry handlers which were each mapped to different
APIDs. A module for each type of handler was implemented that included a
function to register handlers for each type of command, and a function to
dispatch the correct handler according to the input buffer. Each application
used the first byte of the spacepacket data payload as an index into the
maintained map of handlers.

To demonstrate the correct dispatch and execution of these application handlers
over the spacepacket link, a test component
(\lstinline|fsw/src/app/test_component.c|) was implemented that exposed several
actions, parameters and telemetries. The \lstinline|test_component| included
\lstinline|uint32_t| and \lstinline|uint8_t| static variables that were exposed
as parameters, action handlers that would print the values of these parameters
to the debug log, and a telemetry that would return the sum of the two
parameters over the spacepacket link.

In addition, telemetry values were included to monitor runtime error
information in different parts of the application, such as in the
\lstinline|frame_buffer| and \lstinline|spacepacket| modules.

To allow easy extensibility of the FSW, Action, Parameter and Telemetry
handlers were registered on start-up after creating the RTOS threads but before
the RTOS had been executed.

\subsection{Ground Software}

The Ground Software (\lstinline|gsw|) application was simple in comparison to
the flight software. This was a command line application written in python
using the click \citep{click}, pySerial \citep{pyserial} and spacepackets
\citep{spacepackets} libraries. Sub-commands were implemented as click
functions that built correctly formed spacepackets containing Actions,
Parameters and Telemetries using the spacepackets library and sent them to the
target hardware over an FTDI cable using pySerial. Additionally, a monitor
command was written that used pySerial to connect to the debug USART of the
target and print any text it output. The tool used code from  the
\lstinline|pgf| library for correctly generating and appending checksums to
packets, as well as for performing kiss framing of spacepackets. The
\lstinline|gsw| tool was the main interface used for debugging and testing the
FSW on the target hardware.

\subsection{Tools} \label{sec:fsw-tools}

Additional tools and commands were included in the \lstinline|Makefile| to aid
with debugging and testing. These included tools for generating a disassembly
of the compiled FSW; disassembling specific memory addresses or function
symbols; running the FSW on the target under a debugger; and extracting memory
address locations from the compiled binary.

% Discuss tooling, rationale for development decisions?

% Talk through all the tools that you had to develop, explain problems that
% were encountered, and their resolutions, explain bugs found and methodologies
% they were resolved

% overview of tooling developed? Identify dependencies, modified design choices


\section{Emulator}

% section about emulator development
% - unicorn only an ISA emulator
% - discuss implementation of interrupt handling, trampoline handlers

The emulator implementation used the Unicorn framework \citep{Unicorn} to execute the
instructions from the FSW. Most of the emulator implementation effort was spent
modelling hardware devices, and implementing fundamental features. As Unicorn
was only an CPU emulator \citep{Unicorn}, the hardware features of ARM Cortex-M
chips beyond executing instructions had to be implemented manually. These
included context switching between interrupts, reading and writing data over
the USART peripheral to the running firmware, and correctly triggering and
handling the PendSV, SysTick and USART interrupts. Given the comprehensive
understanding of the low level details of the developed FSW, it was easier to
identify the reduced set of peripherals required to be modelled in the
emulator, and implement them incrementally.

Unicorn provided APIs for interacting with the instruction set emulator in
multiple languages, but for this project python was chosen as the
implementation language. It was determined that python and C were the most
commonly used languages for interacting with Unicorn, and state of the art
tools like Fuzzware made use of both \citep{Fuzzware_2022}. Python was chosen
over C due to being more able to quickly prototype and test implementations,
despite potential performance improvements from using a natively compiled
language like C.

The development of the emulator was conducted alongside the FSW, so as features
were added to the FSW, mirrored implementation needed to be added to the
emulator. Initially, the FSW had no RTOS or USART interface, and just
periodically toggled a GPIO. At this point, the development of the emulator
began. The emulator would create the Unicorn object; setup the memory map,
which mirrored the linker script outlined in \autoref{sec:linkerscript}; and
add a code hook that disassembled and printed the currently executed
instruction. This functionality was implemented in an \lstinline|Emulator|
class located in \lstinline|emu/emulator.py|.
This iteration mapped memory regions used for peripherals such as
the RCC and SysTick registers as generic read write memory, so when the
firmware accessed these locations there was no functional impact. As a result,
the periodic toggling of the GPIO pin was not correctly emulated: the emulator
ran the software as if the SysTick peripheral never counted down.

\subsection{Core Peripheral Models}

The next iteration implemented generic peripheral models for MMIO registers.
This model allowed a region of memory to be written too, and the value written
be read from at a later date, using Unicorn MMIO hooks. The MmioReg
(\lstinline|emu/mmio/reg.py|) and Peripheral
(\lstinline|emu/mmio/peripheral.py|) classes provided base implementations that
could be inherited from and have functions overridden to define additional
behaviours. For example, the SysTick Peripheral Model was implemented by
inheriting from the \lstinline|Peripheral| class. The class configured
\lstinline|MmioReg| objects for each SysTick register, with the correct memory
address offsets for each register. The \lstinline|write_cb| method was
overridden to set an enabled flag on the object if the zeroth bit on the
SysTick Control and Status Register was set \citep{armcm4_manual}. This feature
emulated the behaviour of the hardware peripheral, loading the value in the
LOAD register into the VAL register, and enabling the tick method of SysTick
peripheral model. The tick method decremented the value in the VAL register,
and upon reaching zero would reload the value in the LOAD register, and set a
\lstinline|systick_pending| flag on the peripheral model, to indicate to the
emulator that a SysTick interrupt should be triggered.

Other Cortex-M peripheral models were implemented in \lstinline|emu/cortex_m|
according to the Cortex-M4 User Guide \citep{armcm4_manual}. In addition to the
SysTick model, these included the System Control Block (SCB), which included
registers used to raise the PendSV interrupt and set the priority of the
SysTick and PendSV interrupts; and the Nested Vector Interrupt Controller
(NVIC), which was used for enabling and setting the priority of the USART
peripheral interrupts.

These peripheral models were composed into the \lstinline|CorePeripherals|
class in \lstinline|emu/cortex_m/core|. This class setup the Unicorn MMIO
memory map and callbacks for the full address range of all the Cortex-M core
peripherals, and dispatched the callback to the correct peripheral model for
the memory address. Any memory address that did not have a peripheral model
implemented would raise a \lstinline|PeripheralNotEmulatedException|. This
was useful when integrating CMSIS into the FSW, as some core peripherals used
in the CMSIS NVIC API had not been implemented. When the FSW was run on the
emulator, this threw a clear exception that indicated which peripheral models
and register implementations were missing.

\subsection{Interrupt Handling} \label{sec:emu-isr}

The code callback (\lstinline|uc_code_cb|) in the \lstinline|Emulator| class
was used to introspect the modelled peripherals and execute "hardware"
functions, such as triggering interrupts. The callback called the tick method
on the SysTick peripheral model, and checked the \lstinline|CorePeripherals|
object for any pending SysTick or PendSV interrupts. The initial prototype
attempt to make the emulator switch execution to an interrupt handler was to
hard-code the program counter (PC) in Unicorn to the address of the ISR, which
was taken from the disassembled vector table. However, this approach was
unsuccessful.

Similarly to the RTOS context switching implemented in \autoref{sec:fsw-rtos}, ARM
processors followed a defined exception handling sequence \citep{ARM_Exception}
which involved pushing the context of registers onto the stack. However, a
major difference between this process and the AAPCS \citep{AAPCS} is the
contents of the link register (LR). The LR is used in the AAPCS to indicate the
memory location to return to after a function call. For an interrupt, this is
set to the special \lstinline|EXC_RETURN| value, which indicates to the
processor to carry out the return from interrupt sequence.

Initial attempts at returning from interrupts had the code callback checking
for a LR value matching \lstinline|EXC_RETURN|, and restoring the context.
However, interrupts are able to be executed during the execution of lower
priority interrupts, which invalidated this approach.

To solve these issues a design pattern for handling interrupts called
trampolines was used \citep{trampoline}. To implement this pattern in the
emulator, a stack data structure of callbacks was initialised with the
\lstinline|Emulator.reset_handler()| trampoline handler method. A trampoline
handler was defined as a callback function that would start the execution of
the emulator. Therefore, the \lstinline|Emulator.reset_handler()| method would start
the Unicorn emulator execution from the \lstinline|Reset_Handler| in the vector
table. When the \lstinline|Emulator.start()| method was called to begin
emulator execution, it would run an infinite while loop, popping trampoline
handlers from the trampoline stack and executing them. The execution of the
trampoline handlers would continue until an exception was raised or a
trampoline handler returned with no more trampoline handlers left on the
trampoline stack to execute.

While the \lstinline|uc_code_cb| was running, if a peripheral model had flagged an
interrupt as pending, such as PendSV, the
\lstinline|Emulator.handle_interrupt()| method would be called. This method
pushed the \lstinline|return_from_interrupt| and \lstinline|isr_handler| trampoline
handlers onto the trampoline stack, and stopped the execution of the emulator.
This would trigger the next trampoline handler to be executed, which would be
the \lstinline|isr_handler| function.

The \lstinline|isr_handler| trampoline handler saved the current execution
context on the stack according to the ARM Exception Handling Sequence
\citep{ARM_Exception}, set the PC to the address specified in the vector table
for the relevent interrupt, and loaded the \lstinline|EXC_RETURN| value into
the LR. The \lstinline|isr_handler| then restarted the emulator execution from
the updated program counter, with an exit address specified as the
\lstinline|EXC_RETURN| value. Therefore, when the ISR function in the FSW
returned, Unicorn would jump execution to the \lstinline|EXC_RETURN| address,
which would trigger the emulator to stop execution. This in turn would trigger
the execution of the next trampoline handler on the stack, which would be the
\lstinline|return_from_interrupt| function. The
\lstinline|return_from_interrupt| function restored the context stored on the
stack by the \lstinline|isr_handler| function, and restarted the emulator from
the recently restored value of the program counter. In the event that a higher
priority interrupt was triggered, such as the SysTick, while in the context of
a lower priority interrupt, additional trampoline handlers would be added to
the trampoline stack. This ensured that each ISR was correctly executed and the
context of lower priority interrupts restored. The implementation was also
compatible with the FSW RTOS with no additional implementation in the emulator,
which as mentioned in \autoref{sec:fsw-rtos} used the PendSV ISR to save the
context of thread execution on the stack, and swapped the stack pointer and
program counter for the new thread to run.

% difficulties with building emulator and verifying against hardware Discussion?
However, there was a lot of complexity associated with integrating the emulator
with the interrupt execution, context switching, trampoline handlers and
peripheral models. The implementation required a lot of refinement as simple
errors, such as the order of checking the pending interrupts, had dramatic
impacts on the execution of the flight software. Debugging these errors was
difficult, and the execution of the FSW on the emulator was heavily compared to
execution on the target hardware to understand and resolve issues. This process
of using the target hardware to verify the behaviour of the emulator somewhat
invalidated the rationale behind \refrq{1} for being able to incrementally
verify the FSW using emulation in the absence of target hardware.

The core emulator implementation was not concerned with the details of the FSW
application beyond the low level interfaces to the hardware in the HAL and
RTOS. Therefore, once the RTOS based FSW was able to run on the emulator, the
focus shifted to reading and writing data to the running firmware through the
USART peripheral model.

\subsection{USART Peripheral Model}

The USART peripheral was the only STM32 hardware peripheral used by the FSW,
and therefore the only one that needed to be implemented in the emulator. The
implementation of the USART peripheral model is located in
\lstinline{emu/uart.py} as a \lstinline|Uart| class inheriting from the
\lstinline|Peripheral| class. Similar to the SysTick peripheral model, the
\lstinline|Uart| class defined a set of \lstinline|MmioReg| objects for each
USART register as defined in the Reference Manual \citep{stm32f4_manual}.

However, the \lstinline|DATA| register of the USART peripheral required a
specific implementation to model the behaviour of sending and receiving data.
When a value is written to the \lstinline|DATA| register, that value is pushed
to a hardware FIFO to be sent, and when a value is read, it is pulled from a
different hardware FIFO into the register. The \lstinline|UartDataReg| class
inherits from \lstinline|MmioReg| to overwrite the read and write callbacks
to implement this functionality, with read and write FIFOs implemented in
python. Therefore, to "send" data to the emulated FSW, data can be appended to
the modelled FIFOs and the correct bits set in control registers to tell the
software there is data available to be read. The same approach is taken to
"receive" data output from the emulated FSW. The \lstinline|Uart| class provides
methods for reading and writing bytes, or arrays of data, to the peripheral.
After the FSW was updated to implement interrupts for asynchronous reading of
data from the USART \autoref{sec:fsw-hal}, the \lstinline|Uart| object needed
to trigger interrupts when data had been written to the peripheral. The process
for doing so was similar to setting a flag for the SysTick or PendSV to
pending, doing so whenever data was written to the modelled FIFO. However, the
method used for writing data to the \lstinline|Uart| object was to write an
entire buffer into the modelled FIFO. As each byte of data needed a separate
interrupt raised to read it, the interrupt pending flag needed to remain set
until the modelled FIFO was empty.

In order to prevent errors when managing data in the modelled FIFOs, trampoline
handlers were implemented that would stop execution of the Unicorn emulator in
order to print the data they contained. Each \lstinline|Uart| model was
configured with a terminator byte that would trigger the data in the modelled
FIFO to be flagged as ready to print. For the debug USART, this was a newline
character, while for the spacepacket USART, this was a KISS FEND byte.

\subsection{Command Line Interface}

Now the emulator had enough functionality to boot and execute the complete FSW.
To easily run the FSW on the emulator and enable different features for
debugging and testing, a simple command line interface was created. This allowed
running the FSW under the emulator, with raw spacepacket data preloaded into
the emulator before execution, and the responses and debug messages from the
FSW printed to the console. Each raw spacepacket would be loaded and "sent" to
the emulated software sequentially, and the raw spacepacket data included a
"trigger" byte at the beginning of the bytearray, which instructed the emulator
when to load them into the USART peripheral. This setup allowed for simple
manual testing of the FSW on the Emulator, similar to what had been possible
with the target hardware and \lstinline|gsw| program.

\subsection{Debugging Features}

Several features were included in the emulator implementation to aid debugging
crashes or other problems (\refreq{EMU-5}). The Capstone python library was
included in the emulator implementation, which could be used to print
disassemblies of instructions \citep{Capstone}. This functionality was used to
introspect and debug the running of the FSW and the Emulator at various stages
of development. For example, before implementing kiss framing of spacepacket
data, sending spacepackets to FSW running on the emulator resulted in packets
being dropped and other framing issues. However, this was not the case on the
target hardware. To debug this, the memory address ranges for specific
functions in the flight software, such as reading from the uart and processing
the spacepacket, were extracted from compiled FSW binary using tools outlined
in \autoref{sec:fsw-tools}. Statements were added to the
\lstinline|uc_code_cb| function in the emulator to disassemble and print any
instructions executed in those memory addresses. Additionally, logic could be
added to dump specific memory regions and registers during execution of
specific instructions by the emulator. This helped identify where the
spacepackets were being dropped by the FSW running on the emulator. Ultimately,
it was discovered that without framing implemented, spacepackets could be sent
and received successfully only if a single complete spacepacket was received by
the FSW in-between the RTOS context switching threads. The Emulator was sending
spacepackets to the running FSW quicker, relative to the number instructions
executed, compared to sending multiple spacepackets to the target hardware. As
such, running the FSW on the emulator was able to identify the need for framing
of input data, which would have been very difficult with only the target
hardware. Several debug statements in both the FSW and Emulator are present in
the final artefacts, wrapped behind compile time defines and command line
options. However, most debug statements were removed after issues were resolved
and the software reworked.

% CUT This one has been written about in the emulator implementation
% Framing issues when sending data before implementing the kiss framing This was
% discovered when developing the emulator which was sending all the spacepackets
% at the same time, causing multiple packets to get processed as one, and
% indicating the need for framing. Initially, some fixes were implemented in the
% spacepacket process function, before properly implementing KISS framing

\section{Protocol Grammar Filter}

The emulator could be run with manually written valid binary strings
representing kiss encapsulated space packets. However, to generate valid input
data from a random binary stream (\refreq{RQ2-3}), the \lstinline|pgf| python
library was implemented.

The main class in the \lstinline|pgf| library was \lstinline|PacketStream|. The
\lstinline|PacketStream.from_bytestream()| static method was a python generator
that lazily converted a variable length iterable bytearray object into a stream
of \lstinline|PacketStream| objects. Each \lstinline|PacketStream| object used
the input bytes to build a spacepacket containing application data and a
checksum, encapsulated in a kiss frame. The \lstinline|PacketStream| constructor
pulled bytes from the input bytestream, and used them to determine which errors
to inject into the output data. The \lstinline|PacketStream.to_bytestream()|
method was a generator that lazily created a stream of protocol data from the
\lstinline|PacketStream| object.

The PGF used valid constant values for most fields in the spacepacket header,
but selecting the APID and the ID of the application handler used bits from the
input data stream. A configuration byte in the input data indicated which type
of error should be injected into the output, or if the output should be well
formed. These errors included stopping the packet from being encapsulated in a
KISS frame; invalidating the calculated checksum; Setting specific fields in
the spacepacket header to invalid values; and invalidating the application
data.

The PGF was integrated into the emulator through the implementation of the
\lstinline|SpacepacketHandler| class, which accepted either a raw protocol
datastream or PGF input datastream, and output a stream of packets to the
emulator with time delays according to trigger values.

% design the input format
% outline the output
% outline the set of known errors based on spacepacket fields

\section{Black Box Fuzzer}
Blackbox fuzzing was implemented for both the \lstinline|gsw| tool and the
emulator. The blackbox fuzzer was a python generator that provided an infinite
stream of random bytes. For the \lstinline|gsw|, this could be used as the input
bytestream to the PGF, in a function called \lstinline|grammarstream|.
Alternatively, the random data could be used as raw data to be sent, using a
utility function \lstinline|raw_input_stream|, which used the first byte of
data in the stream to determine the trigger and length of input data to send.
Now both raw and grammar blackbox fuzzing could be conducted against the target
hardware.

The emulator implemented blackbox fuzzing in an equivalent manner, using the
\lstinline|blackbox_generator| function as an input to the
\lstinline|SpacepacketHandler|, which could use it as either raw or grammar
input depending on command line options.

Due to the simplicity of the blackbox fuzzer, using the raw input was not very
useful. As there was no coverage feedback, most of the input data generated by
the raw input was rejected by the flight software at the first condition.
However, using the blackbox generator in conjunction with the PGF provided
useful inputs almost all the time, and was very effective for testing on both
the emulator and the hardware.

\section{Coverage Guided Fuzzer}

The AFL++ coverage guided fuzzer \citep{AFLplusplus} was integrated with the
implemented Unicorn emulator using UnicornAFL \citep{UnicornMode}.

% modification to emulator to get it to work as a afl++ test harness?
The main modifications to the emulator were the propagation of errors,
discussed in \autoref{sec:fuzz-err}; and several optimisations to improve the
performance of the emulator, like removing unnecessary print statements.
Furthermore, a handler function was added to the emulator command line
application that would initialise the \lstinline|Emulator| object, setup a
callback function for the fuzzer to provide input, and call the
\lstinline|uc_afl_fuzz_custom| function. Due to the implementation of
trampoline handlers (\autoref{sec:emu-isr}), the standard
\lstinline|uc_afl_fuzz| function could not be used as this function triggered
\lstinline|Uc.emu_start| and exited when the emulation stopped.  The
\lstinline|uc_afl_fuzz_custom| function allowed passing a custom callback to
run the emulator, which allowed for the trampoline handlers to stop and start
the Unicorn emulation.

% using docker to run AFL
The documentation for AFL++ recommended using a provided Docker image to run
the fuzzer \citep{AFLDocker}. However, this Docker image did not include the
dependencies required by the emulator and PGF. Therefore, the AFL++ image was
extended to include those dependencies in the \lstinline|Dockerfile|. Additionally,
commands for running AFL++ with the emulator, as well as building and launching
the docker container, were added to the \lstinline|Makefile|.

\subsection{Error Handling} \label{sec:fuzz-err}

In order run the FSW with UnicornAFL, different termination conditions needed
to be mapped to UnicornAFL errors. The emulator was modified to raise python
exceptions, and the \lstinline|fuzz_start| method would catch those exceptions
and handle them appropriately. To prevent every correct input resulting in a
timeout, the \lstinline|OutOfPacketsException| was raised a short period after
all spacepackets had been sent to the FSW. This exception was handled to exit
gracefully with no UnicornAFL error, as it was assumed any crashing input would
have already failed by the time the exception was raised. Alternatively, the
\lstinline|DbcException| was raised when the emulator detected the FSW had
entered the \lstinline|DBC_fault_handler|. This outcome was considered a crash
of the FSW, and so would trigger a UnicornAFL error.

\subsection{Example Data Generation}

AFL++ required a directory of files which were known
good inputs to the application being fuzzed as a starting point for generating
input data. A python script was written to generate valid input data for both
the PGF and raw fuzzing runs
(\lstinline|tools/generate_inputs.py|).

\end{document}
