\documentclass[../report.tex]{subfiles}
\graphicspath{{\subfix{../images}}}
\begin{document}

%% Software Design?

% [ ] Outline software design needed to meet requirements
% - list all software developed, why they are needed, what requirement they satisfy
% - explain design decisions, such as STM32, DBC, minimal rtos, unicorn
% [ ] Outline tasks needed to meet requirements

% Software tools required:
%  * ground software:
%   ** send and receive spacepackets
%   ** blackbox fuzzer
%    *** grammar filter
%    *** raw data
%  * rehosting software (emulator)
%   ** interrupt handling

All the software implemented is included in a single repository

% Implementation (and Testing?)
\section{Flight Software}

\subsection{Implementation Decisions}

The most logical and popular choice of programming language for embedded flight
software is the C programming language. C is used in FSW frameworks like Gen1
and cFS \citep{Bal_FSDK, Nasa_cFS}. Languages like C++ and Ada are also
commonly used, C++ is used in the F Prime FSW framework \citep{Nasa_fprime}.
Recently, modern languages like Micropython, Rust and Zig have started to
become popular for embedded devices. Rust was used by the University of
Stuttgart for their sat-rs flight software library, which flew as an experiment
on the ESA OPS-SAT satellite \citep{satrs, opsat}. However, support for these
modern languages is still in progress, and currently do not have the stability
required for commercial projects. In preparing for developing the flight
software, an initial bare metal program for the STM32 BlackPill was built using
Zig, but it was decided that using Zig over C would add unneccesarry complexity
and reduce the representativeness of the FSW.

The FSW made use of design by contract as a methodology for ensuring the
correctness of the implementation. % TODO

Requirement \refreq{FSW-9} specifies the need for an RTOS in the FSW, to make
the implementation representative. An alternative to using an RTOS is the
"superloop" architecture, where all execution happens in a single while loop.
This architecture is also commonly used in flight software, but mostly for
Telemetry Nodes, which only need to execute a small number of functions. An OBC
software typically needs to be able to execute multible threads concurrently,
which cannot be done using a superloop architecture. There are alternative
concurrent architectures, such as the active object based QP framework
\citep{QP}. However, RTOS is the most commonly used design.

Common RTOS implementations used in flight software include FreeRTOS, RTEMS and
VxWorks \citep{Cratere_2024}. FreeRTOS is open source, and so the use of this
was investigated for the project. However, rather than use an existing RTOS
implementation, it was ultimately decided to develop a simple multitasking RTOS
for the STM32 Blackpill, based on the educational RTOS MiROS \citep{miros}.
While this was less representative, the decision increased the simplicity of
the FSW, and developing the RTOS using educational resources would make the
implementation of emulator features required to support the RTOS much easier.

The hardware abstraction layer, STM32 HAL, CMSIS

The application code

% - Written in C? The only real choice
% - Bare Metal
%   - HAL
%   - Startup Code
%   - Design by Contract
% - Actions, Parameters, Telemetries, RPC
% - Spacepacket Interface
% - Bare Metal RTOS inspired by MIROS (rather than FreeRTOS or RTEMS, easier to reason about and emulate for the demonstration of the project)
% - CMSIS? Avoid using relying on CMSIS to reduce the amount of perhipherals and mmio that would need to be emulated.

\subsection{Implementation Details}


% Implemented in the C language

% Discuss tooling, rationale for development decisions?

% Talk through all the tools that you had to develop, explain problems that
% were encountered, and their resolutions, explain bugs found and methodologies
% they were resolved.

% overview of tooling developed? identify dependencies, modified design choices, not implementing space data link layer.

% brief overview of iterative development process, with features implemented between tooling alongside each other

% section about flight software
% linker scripts, makefiles, writing the hal
% utils

\section{Emulator}

% section about emulator development
% - unicorn only an ISA emulator
% - discuss implementation of interrupt handling, trampoline handlers


% TODO this can go in results/testing/discussion section?
% section about testing on hardware, and bugs discovered, addition of framing and


\section{Fuzzer Integration and Protocol Grammer Filter}

% modification to emulator to get it to work as a afl++ test harness?
% using docker to run AFL

% FIXME Discussion?
As the PGF can only introduce known errors into the protocol, the design of the
PGF is a compromise between comprehensive fuzzing and specific manually
designed system testing.

% design the input format
% outline the output
% outline the set of known errors based on spacepacket fields

\section{FIXME REMOVE}
% DEMO TEXT BELOW
This is the chapter in which you review the implementation and testing decisions and issues, and critique these processes.

Code can be output inline using \verb@\lstinline|some code|@.  For example, this code is inline: \lstinline|public static int example = 0;| (we have used the character \verb@|@ as a delimiter, but any non-reserved character not in the code text can be used.)

Code snippets can be output using the \verb|\begin{lstlisting} ... \end{lstlisting}|
environment with the code given in the environment. For example, consider listing \ref{Example-Code}, below.

\begin{lstlisting}[breaklines,breakatwhitespace,caption={Example code},label=Example-Code]
public static void main() {

  System.out.println("Hello World");

}
\end{lstlisting}

Code listings are produced using the package `listings'.  This has many useful options, so have a look at the package documentation for further ideas.

\end{document}
