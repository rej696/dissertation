\documentclass[../report.tex]{subfiles}
\graphicspath{{\subfix{../images}}}
\begin{document}

% Implementation (and Testing?)

% Discuss tooling, rationale for development decisions?

% Talk through all the tools that you had to develop, explain problems that
% were encountered, and their resolutions, explain bugs found and methodologies
% they were resolved.

% overview of tooling developed? identify dependencies, modified design choices, not implementing space data link layer.

% brief overview of iterative development process, with features implemented between tooling alongside each other

% section about flight software
% linker scripts, makefiles, writing the hal
% utils

% section about emulator development
% - unicorn only an ISA emulator
% - discuss implementation of interrupt handling, trampoline handlers

% section about testing on hardware, and bugs discovered, addition of framing and

% DEMO TEXT BELOW
This is the chapter in which you review the implementation and testing decisions and issues, and critique these processes.

Code can be output inline using \verb@\lstinline|some code|@.  For example, this code is inline: \lstinline|public static int example = 0;| (we have used the character \verb@|@ as a delimiter, but any non-reserved character not in the code text can be used.)

Code snippets can be output using the \verb|\begin{lstlisting} ... \end{lstlisting}|
environment with the code given in the environment. For example, consider listing \ref{Example-Code}, below.

\begin{lstlisting}[breaklines,breakatwhitespace,caption={Example code},label=Example-Code]
public static void main() {

  System.out.println("Hello World");

}
\end{lstlisting}

Code listings are produced using the package `listings'.  This has many useful options, so have a look at the package documentation for further ideas.

\end{document}
