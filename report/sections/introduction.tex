\documentclass[../report.tex]{subfiles}
\graphicspath{{\subfix{../images}}}
\begin{document}

% short background, modivation, Outlining of Research Questions
% backwards conclusion, saying what we want to find out, what the expectations are the only real difference is the tense
% don't make it too long

% [ ] Explanation of developing software in new space
% [ ] Explanation of Emulation and Fuzzing
% [ ] Research Questions and Objective

% RQ1: Analyse the effectiveness of different fuzzing methods for incremental development of flight software
% RQ2: Analyse the impact of using a protocol grammar filter on fuzzer performance, measured in coverage

\section{Research Questions} \label{sec:rq}
This project aims to analyse the effectiveness of using rehosting techniques
for coverage guided fuzz testing of embedded software in the context of space
flight software development. The work will investigate the state of the art in
embedded software fuzz testing, and apply these techniques during the
development of a representative flight software program, in order to determine
if the use of fuzz testing and emulation can improve code quality, reduce costs
and increase productivity for software developed in the New Space industry. Any
difficulties or concerns arising as a result of completing the work will be
raised and discussed.

Furthermore, the use of a protocol grammar filter for rehosted coverage guided
fuzzing of embedded software will be evaluated. A simple filter will be
developed and used to fuzz test the developed flight software. The results will
be compared against the same fuzzer, without the protocol grammar filter, and
any findings and observations will be discussed.

Therefore, the work will aim to answer the following two research questions:

\begin{description}
    \item[\definerq{1}]Analyse the effectiveness of different fuzzing methods for incremental development of flight software
    \item[\definerq{2}]Analyse the impact of using a protocol grammar filter on fuzzer performance, measured in coverage

\end{description}

\section{Motivation} \label{sec:motivation}

% Rationale/Motivation:
% - Comprehensive testing without hardware?
% - Security threats over space links?
% - Difficulty of coverage guided fuzzing on embedded targets
%   - Identified future work from hoedur paper regarding protocol fuzzer and streams (explain more in literature review)

% References:
% - Bousedra_2024: policy paper, may be less useful?
% - Denis_2020: paper discussing trends in the space industry, more relevent?
% - Cubesat_Handbook: book with information about designing cubesats, useful for satellite architecture?
% - Cratere_2024: OBC for Cubesats State of the Art

Historically, designing, manufacturing and operating space missions has been a
costly undertaking, with a large number risks and specific challenges. Software
systems in spacecraft have been carefully designed over long time periods to
ensure that there was the possibility of failure was minimal, due to the high
cost and difficulty with trying to resolve issues after launch.

However, the cost of launches to low earth orbit has been reducing, driven by
the commercialisation of the space sector and prolific companies like SpaceX
\citep{Denis_2020}. This, and the development and proliferation of the CubeSat
standard \citep{CubesatDesignSpec}, has enabled lower cost space missions with
shorter development times.

The software written for CubeSats typically runs on embedded microcontrollers,
and has a high level of interaction with hardware interfaces
\citep{Cratere_2024}. The development of software for these systems therefore
often has a critical dependency on hardware. However, hardware designs are
subject to the same cost and time constraints as software designs. This can
lead to prototype hardware being unavailable (or only available for a short
time) for development and testing of software.

Developing software for CubeSats under budget and on time also requires cost
effective verification techniques. Fuzz testing has been effectively used to
verify non-embedded software. For example, Google has used fuzz testing to
detect over 10000 security vulnerabilities in over 1000 open source projects
through its OSS-Fuzz project \citep{Google_2023}. State of the art coverage
guided fuzzers typically rely on using operating system features to instrument
and test target software applications for general purpose computers. Embedded
devices typically use different processor architectures and instruction sets,
and do not include the operating system features required for coverage guided
fuzzers to run natively \citep{Muench_2018}.

Emulating embedded software on general purpose computers has become the most
popular methodology for fuzz testing embedded systems \citep{Yun_2022}. The
process of using an emulator to host an embedded software program to run on a
general purpose computer with a fuzzer is often called "rehosting". The variety
in architectures, operating systems, and hardware integration in embedded
software makes a general purpose solution difficult to achieve. However, there
have been several developments trying to improve the process of fuzz testing
rehosted embedded software. One of the most recent developments is called
Hoedur, which improves the interface between the emulation software and fuzzer,
such that each virtual memory location uses a separate input data stream
\citep{Hoedur_2023}. \citet{Hoedur_2023} posit that further improvements to the
fuzzer could be made if specialised grammar fuzzers were used for input streams
corresponding to embedded system communication interfaces.

\end{document}
