\documentclass[11pt]{article}
\usepackage[utf8]{inputenc}
\usepackage{graphicx}
% \usepackage[figuresleft]{rotating}
\usepackage{pdflscape}
\usepackage{pdfpages}
\usepackage{float}
\usepackage{caption}
\usepackage{subcaption}
\usepackage{longtable}
\graphicspath{ {images/} }

\setlength{\parindent}{0em}
\setlength{\parskip}{1em}

% Set up nicer table spacing
\setlength{\arrayrulewidth}{0.5mm}
\setlength{\tabcolsep}{18pt}
\renewcommand{\arraystretch}{1.5}

\usepackage{natbib}
\newcommand*{\urlprefix}{Available from: }
\newcommand*{\urldateprefix}{Accessed }
\bibliographystyle{bathx}

\usepackage{geometry}
\geometry{
  a4paper,
  total={170mm,257mm},
  left=20mm,
  top=20mm,
}
% \geometry{
%   a4paper,
% }
\DeclareCaptionFormat{custom}
{%
    \textbf{#1#2}#3
}
\captionsetup{format=custom}
\captionsetup{width=0.8\textwidth}

\usepackage{csquotes}
\usepackage[hidelinks]{hyperref}
\newcommand{\refreq}[1]{\textbf{\hyperlink{req#1}{UR#1}}}
\newcommand{\definereq}[1]{\hypertarget{req#1}{UR#1}}

% Gantt Chart package
\usepackage{pgfgantt}

\usepackage{subfiles}

\title{An Analysis of Emulation based Fuzz Testing for NRF52 Embedded Software Verification}
\author{Rowan Saunders}
\begin{document}

% A Comparison of Fuzzing-based Verification techniques for the development of NRF52 based embedded systems

% - Emulation: coverage-guided unicorn-afl fuzzing
% - Target: Black-box fuzzing
% - Target: coverage-buided Trace based fuzzing
% - Target/Emulation: Manual Test Scripts

\maketitle

% Title
% - Detailed description of the problem you will seek to address in the project and methodologies (3 pages)
% - Objectives and deliverables for the project (1 - 2 pages)
% - Detailed project plan using appropriate techniques (Gantt Chart) (1-3 pages)
% -- Tasks to undertake, time allocation for the tasks, and dependencies
% - List of required resources to complete the project, their availibility, and alternatives if not available.
% - Attach Ethics Checklist

% TODO:
% - Ethics Checklist (Complete)
% - Read methodology content
% - clarify approach referencing Borsig_2020 esp32 but with nrf52. plus focus on emulation using unicorn.
% - Read Beckmann_2023 (plus 10 other relevent papers)
% - Gantt Chart

% Brief
% - Description of the problem you will seek to address in the project (1 to 2 pages plus references)
% - Identification of the main objectives and deliverables for the project (one page)
% - An outline of the project plan (one page) that identifies the tasks you will undertake and the initial time allocation for those tasks.

% Introduction
% - What is Fuzzing?
% -- Existing Uses: Cybersecurity, systems programs (Browsers)
% -- Comparison with other verification methodologies
% - What is Embedded Software (IOT)?
% - Why Fuzzing for embedded software?
% -- Context for Embedded Software development
% -- Current state of embedded software fuzzing (Yun_2022, Eisele_2022)
% --- Outline from these papers areas to work in (coverage guided fuzzing on target or with emulation)
% -- Challenges with fuzzing embedded systems
% --- Constraints of testing embedded software
% ---- Time
% ---- Money
% -- Cybersecurity risks in IOT
% --- Use of insecure languages with memory errors (C/C++)
% - Introduce NRF52 as common chip for wireless applications
% -- Beckmann_2023 uses simple baremetal NRF52 application (and complex rtos on STM32) using hardware coverage guided testing over SWO interface.



% - Difficulty Testing Embedded Software, IOT, etc.
% - What is Fuzzing? Fuzz testing. Existing Use cases
% - Minimal use of fuzzing for testing IOT/Embedded devices
% - Challenges specific to fuzzing embedded devices
%   - Coverage/Feedback to the Fuzzer
%   - Interfaces/Harnesses
%   - Emulation/Use of hardware

% Identify the Data needed to demonstrate project objectives have been achieved.
% Use a project data table:
% - Objective, How will you prove the objective is achieved, What data is necessarry, How can you obtain the data.
% - | Objective | Proof of achievement? | What data needed? | Method to obtain data? |

% Show a process for fuzz testing the NRF52
% - Develop a Unicorn-AFL based test harness for fuzz testing the example application
% - Develop a Unicorn Emulator for targeted (manual) testing of example application
% - Disscus viability of manual testing vs fuzz testing

% Emulate NRF52 using Unicorn
% Integrate NRF52 Unicorn Emulation with AFL++ unicorn_mode
% Build on work done in https://github.com/befoulad/nrf52_radio_emu (emulating nrf52 to run bare metal radio firmware)

% Maier_Unicorefuzz makes use of unicorn emulation to fuzz test an operating system kernel
% README for afl++ unicorn_mode https://github.com/AFLplusplus/AFLplusplus/blob/stable/unicorn_mode/README.md
% FIRMCORN builds on unicorn to develop their own fuzzer that detects function calls

% - Detailed project plan using appropriate techniques (Gantt Chart) (1-3 pages)
% -- Tasks to undertake, time allocation for the tasks, and dependencies
% - List of required resources to complete the project, their availibility, and alternatives if not available.

% - Pilot study
% -- Simple system (few to no peripherals to model in unicorn)
% -- identify test corpus
% -- identify manual testing method (emulation)
% -- identify manual testing methodd (on-target)
% -- identify black-box testing method

\section{Abstract}

\section{Acknowledgements}

% contents page

\section{Introduction}

\subfile{sections/introduction}

\section{Description of Work} \label{sec:1}

\section{Analysis}

\section{Conclusions and Future Work}




% Research Question
% 1. Verification of Real Time Embedded Systems through Fuzz Testing: A Case Study
% 1.1 What Embbeded System?
% - Simple IOT data logger
% - Simple Robot with a couple of sensors and an actuator
% - Synth/dsp device?
% 2. Replacing unit tests with fuzz testing: Automated Software Verification of an Embedded System.

\section{Project Objectives and Deliverables} \label{sec:2}
This research project aims to investigate the perceived difficulty of fuzzing
an embedded system with currently available tooling, and make a comparison
between different fuzzing and other verification methods with regards to that
difficulty. The project will outline the procedure for configuring and
executing the different verification methodologies, and gather data about the
effectiveness, time/effort, and reuseability. The objective of the study is to
prove that applying fuzz testing methodologies during embedded systems
development provides a strong benefit, despite any time/effort costs. The
expectation is that fuzz testing may outperform traditional verification
methods with respect to cost, such as manual systems testing, and prove to have
higher reuseability.

To do this, an example bare-metal embedded system will be developed based on
the NRF52 chip, with sufficient features to be representative of a real device.
Test harnesses required for manual and fuzz testing will be developed, and the
time and tasks to produce these will be tracked. The example firmware will then
be tested with each verification method, and the number of errors discovered,
and time taken to discover them, will be recorded. These quantitive data will
be compared and discussed, along with qualititative anecdotal information
regarding challenges encountered with each proposed method.

% The main objective of the study is to measure and compare the
% cost-effectiveness of emulation based coverage guided fuzzing as a verification
% methodology for an IOT bare metal embedded system with other verification
% methodologies, such as manual testing, black-box fuzzing, and on-target
% coverage guided fuzzing.
% Cost-effectiveness will be explored through measuring
% the time and difficulty of implementing and executing each verification
% technique, alongside the quantity and type of errors identified by the method.

To meet this objective, and based on the work outlined in \textit{Section
\ref{sec:1}}, a unicorn based emulator for the example NRF52 based system will
be developed, with test harnesses for fuzzing with AFL++ UnicornMode. Required
NRF52 hardware peripherals will be modeled in the unicorn based emulator.
Manual test scripts derived from software requirements for the embedded system
will be created, and also run on the emulator. Furthermore, a simple black box
fuzzer will be developed to run on the emulator. The core deliverables will be:

\begin{itemize}
\item A literature review outlining and comparing embedded fuzz testing methodologies
\item The design and implementation of an example NRF52 based embedded system
\item A set of system test scripts for manually testing the example system
\item The implementation of a unicorn NRF52 emulator, with appropriate peripherals modeled
\item A set of error reports and bug investigations resulting from testing the example system with a black-box fuzzer on the unicorn emulator
\item A set of error reports and bug investigations resulting from running the test scripts against the example system on the unicorn emulator
\item A set of error reports and bug investigations resulting from testing the example system with unicorn-afl
\item A discussion comparing each emulation based verification methodology, considering cost effectiveness
\end{itemize}

The main objective outlined will be supplemented with additional extension
objectives. One potential extension to the main objective would include an
additional investigation and comparison with hardware based verification
techniques. Coverage guided fuzzing based on Beckmann's tracing based work
\citep{Beckmann_2023} could be implemented, along with developing a test
harness to execute the manual test scripts and apply the black-box fuzzer on
the target hardware. The deliverables for this extended objective would be as follows:

\begin{itemize}
\item The development of test harnesses for on target verification of the embedded system
\item Tooling, Instrumentation and Harnesses required for on target coverage guided fuzzing of the example system
\item A set of error reports and bug investigations resulting from testing the example system with a black-box fuzzer on the target hardware
\item A set of error reports and bug investigations resulting from running the test scripts against the example system on the target hardware
\item A set of error reports and bug investigations resulting from testing the example system with a coverage guided hardware fuzzer
\item A discussion comparing each hardware based verification methodology, considering cost effectiveness
\end{itemize}

An alternative extension objective would be to measure the effect of system
complexity on the cost effectiveness of each verification methodology.
The example system can be updated to use common IOT peripherals, such as wifi
or bluetooth, and model these in the unicorn emulator. The ease of emulating
and fuzzing an embedded system may be correlated with the complexity of that
system, and using a wireless protocol should improve the relevence of the
results of the study to IOT devices. The deliverables for this extended objective would be as follows:

\begin{itemize}
\item The design and implmentation of an expanded NRF52 based embedded system, including the use of a wireless communication peripheral
\item The implementation of a unicorn NRF52 emulator with wireless peripherals modeled
\item A set of expanded system test scripts for manually testing the updated example system
\item A set of error reports and bug investigations resulting from testing the expanded example system with a black-box fuzzer on the unicorn emulator
\item A set of error reports and bug investigations resulting from running the test scripts against the example system on the unicorn emulator
\item A set of error reports and bug investigations resulting from testing the expanded example system with unicorn-afl
\item A discussion comparing each emulation based verification methodology, considering cost effectiveness and software complexity
\end{itemize}


\pagebreak
\section{Project Plan} \label{sec:3}

\textit{Figure \ref{fig:top}} shows a top level gantt chart for the duration of
the project. The timeframe in \textit{Figure \ref{fig:top}} is scoped in terms
of week increments (i.e. 5 working days each). The expected duration of the
project of 13 working weeks is equivilent to 65 working days. Given an expected
average working rate of 8 working days per calendar month, an overall estimate
of the expected timeline can be calculated at around 8 months. This aligns with
the chosen completion period.

The project will initially begin with an extended literature
review, gathering and investigating more literature and developing on ideas
outlined in \textit{Section \ref{sec:1}}. This will lead into a pilot study,
where a basic version of the study will be completed. A detailed work breakdown
for the pilot study is shown in \textit{Figure \ref{fig:pilot}}.

% TODO explain content of pilot study
The focus of the pilot study will be a comparison of emulation based fuzzing
with manual testing and black box fuzzing techniques. The initial example system will
be a simplistic IOT device, using few hardware peripherals. A serial interface
will be used to communicate to the device, which will be simple to integrate
with the unicorn-afl fuzzer.

The work conducted in the pilot study will then be revised and built upon for a
more indepth extended study. There are several potential avenues for
investigation in the extended study, and the direction taken at this point will
depend on the output of the literature review and pilot study. The extended
study will aim to meet at least one of the extension objectives in
\textit{Section \ref{sec:2}}. Two alternative detailed plans are shown for the
extended study in \textit{Figure \ref{fig:ext1}} and \textit{Figure
\ref{fig:ext2}}.

The extended study detailed in \textit{Figure \ref{fig:ext1}} expands the
firmware developed in the pilot study to use more peripherals. The serial
interface will be modified to use a wireless protocol, such as wifi or
bluetooth, which is more common for IOT devices. Comparing test methodologies
for a simple and a more complex system may also show that cost effectiveness of
a test methodology is a function of cost effectiveness, as well as exploring
the ease of modification and reusability of the emulator based test
techniques.

The extended study detailed in \textit{Figure \ref{fig:ext2}} configures a
suite of hardware based verification techniques with the example system
developed in the pilot study. It is anticipated that tooling and harnesses
developed in the pilot study for emulated black-box and manual testing will be
able to be easily adapted to their hardware based equivilents. As such, the
main focus of this work will be on implementing the coverage guided hardware
fuzzing tooling and instrumentation. Currently, time estimates for these tasks
are somewhat uncertain, and so performing the coverage guided hardware fuzzing
last out of the three techniques will at least allow a comparison with the
hardware based manual and black box fuzzing if the coverage guided fuzzing
cannot be completed in the time frame.

% TODO write about each alternative extension (perhaps in objectives section?):
% - Emulation of Bluetooth system for fuzzing
% - Compare Target hardware fuzz testing with emulation fuzz testing

% TODO: List of required resources to complete the project, their availibility, and alternatives if not available.
The work outlined in this plan can be completed with a linux based computer.
The outlined work can be fully completed with open source tools, and where
tooling is not available, this will be created as part of the study. Running
the fuzz tests may require leaving a computer for an extended period, or may
require access to a computer with more powerful hardware (e.g. more RAM). If
this is the case, a Virtual Private Server (VPS) can be purchased with the
required capability for executing the fuzz tests. Some tasks in the extended
study may require access to an NRF52 development kit. This has already been
procured, as it may also be useful for experimenting with during the pilot study.

\pagebreak
\bibliography{report.bib}


\end{document}
